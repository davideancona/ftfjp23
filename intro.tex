\section{Introduction}

Most mainstream object-oriented languages provide a notion of equality between objects which can be customized to be weaker than
reference equality, and which is coupled with the customizable notion of object hash code \cite{Bloch18}. Such two notions are provided
through two corresponding methods defined in the predefined class \lstinline{Object} class which is a the root of the inheritance hierarchy; hence,
they are inherited or redefined in any class, and callable on any type of object.

For this reason, they are pervasive in object-oriented code and the correct functioning of some functionalities of many libraries rely on them;
hence, their incorrect redefinition or use may have a serious impact on software reliability and safety.

A classical example of useful redefinition of equality is for value classes, where typically a notion of logical equality is needed which differs
from reference equality \cite{Bloch18}. Since such classes are immutable by definition, redefining equality is less problematic, although
obeying the general contract for equality is challenging, and its redefinition invalidates the general contract for computing object hash codes. 
In particular, implementations of hash tables are based 
