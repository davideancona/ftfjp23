%%
%% This is file `sample-sigconf.tex',
%% generated with the docstrip utility.
%%
%% The original source files were:
%%
%% samples.dtx  (with options: `sigconf')
%% 
%% IMPORTANT NOTICE:
%% 
%% For the copyright see the source file.
%% 
%% Any modified versions of this file must be renamed
%% with new filenames distinct from sample-sigconf.tex.
%% 
%% For distribution of the original source see the terms
%% for copying and modification in the file samples.dtx.
%% 
%% This generated file may be distributed as long as the
%% original source files, as listed above, are part of the
%% same distribution. (The sources need not necessarily be
%% in the same archive or directory.)
%%
%%
%% Commands for TeXCount
%TC:macro \cite [option:text,text]
%TC:macro \citep [option:text,text]
%TC:macro \citet [option:text,text]
%TC:envir table 0 1
%TC:envir table* 0 1
%TC:envir tabular [ignore] word
%TC:envir displaymath 0 word
%TC:envir math 0 word
%TC:envir comment 0 0
%%
%%
%% The first command in your LaTeX source must be the \documentclass
%% command.
%%
%% For submission and review of your manuscript please change the
%% command to \documentclass[manuscript, screen, review]{acmart}.
%%
%% When submitting camera ready or to TAPS, please change the command
%% to \documentclass[sigconf]{acmart} or whichever template is required
%% for your publication.
%%
%%
\documentclass[sigconf,screen]{acmart}
\usepackage{listings}
\usepackage{js}
\usepackage{js}
\usepackage{xspace}
\usepackage{hyperref}
\usepackage{cleveref}
\usepackage{microtype}
\usepackage{balance}

\newcommand{\rml}{RML\xspace}

\lstset{language=Java,basicstyle=\ttfamily\footnotesize,commentstyle=\itshape,morekeywords={assert},keywordstyle=\ttfamily\bfseries}
%%
%% \BibTeX command to typeset BibTeX logo in the docs
\AtBeginDocument{%
  \providecommand\BibTeX{{%
    Bib\TeX}}}

%% Rights management information.  This information is sent to you
%% when you complete the rights form.  These commands have SAMPLE
%% values in them; it is your responsibility as an author to replace
%% the commands and values with those provided to you when you
%% complete the rights form.


\setcopyright{acmlicensed}
\acmPrice{15.00}
\acmDOI{10.1145/3605156.3606452}
\acmYear{2023}
\copyrightyear{2023}
\acmSubmissionID{isstaws23ftfjpmain-id10-p}
\acmISBN{979-8-4007-0246-4/23/07}
\acmConference[FTfJP '23]{Proceedings of the 25th ACM International Workshop on Formal Techniques for Java-like Programs}{July 18, 2023}{Seattle, WA, USA}
\acmBooktitle{Proceedings of the 25th ACM International Workshop on Formal Techniques for Java-like Programs (FTfJP '23), July 18, 2023, Seattle, WA, USA}
\received{2023-05-26}
\received[accepted]{2023-06-23}

%%
%% For managing citations, it is recommended to use bibliography
%% files in BibTeX format.
%%
%% You can then either use BibTeX with the ACM-Reference-Format style,
%% or BibLaTeX with the acmnumeric or acmauthoryear sytles, that include
%% support for advanced citation of software artefact from the
%% biblatex-software package, also separately available on CTAN.
%%
%% Look at the sample-*-biblatex.tex files for templates showcasing
%% the biblatex styles.
%%

%%
%% The majority of ACM publications use numbered citations and
%% references.  The command \citestyle{authoryear} switches to the
%% "author year" style.
%%
%% If you are preparing content for an event
%% sponsored by ACM SIGGRAPH, you must use the "author year" style of
%% citations and references.
%% Uncommenting
%% the next command will enable that style.
%%\citestyle{acmauthoryear}

%%
%% end of the preamble, start of the body of the document source.
\begin{document}

%%
%% The "title" command has an optional parameter,
%% allowing the author to define a "short title" to be used in page headers.
\title{Runtime Verification of Hash Code in Mutable Classes}

%%
%% The "author" command and its associated commands are used to define
%% the authors and their affiliations.
%% Of note is the shared affiliation of the first two authors, and the
%% "authornote" and "authornotemark" commands
%% used to denote shared contribution to the research.
\author{Davide Ancona}
\email{davide.ancona@unige.it}
\orcid{0000-0002-6297-201}
\author{Angelo Ferrando}
\email{angelo.ferrando@unige.it}
\orcid{0000-0002-8711-4670}
\author{Viviana Mascardi}
\email{viviana.mascardi@unige.it}
\orcid{0000-0002-2261-9926}

\affiliation{%
  \institution{DIBRIS, Universit\`a di Genova}
%  \streetaddress{P.O. Box 1212}
 % \city{Dublin}
 % \state{Ohio}
  \country{Italy}
 % \postcode{43017-6221}
}

%%
%% By default, the full list of authors will be used in the page
%% headers. Often, this list is too long, and will overlap
%% other information printed in the page headers. This command allows
%% the author to define a more concise list
%% of authors' names for this purpose.
%%\renewcommand{\shortauthors}{Ancona et al.}

%%
%% The abstract is a short summary of the work to be presented in the
%% article.
\begin{abstract}
Most mainstream object-oriented languages provide a notion of equality between objects which can be customized to be weaker than
reference equality, and which is coupled with the customizable notion of object hash code.
This feature is so pervasive in object-oriented code that incorrect redefinition or use of equality and hash code
may have a serious impact on software reliability and safety.

Despite redefinition of equality and hash code in mutable classes is unsafe,
many widely used API libraries do that in Java and other similar languages.
When objects of such classes are used as keys in hash tables, programs may exhibit unexpected and unpredictable behavior.
In this paper we propose a runtime verification solution to avoid or at least mitigate this issue.

Our proposal uses \rml, a rewriting-based domain specific language for runtime verification
which is independent from code instrumentation and  the programming language used
to develop the software to be verified. 
\end{abstract}

%%
%% The code below is generated by the tool at http://dl.acm.org/ccs.cfm.
%% Please copy and paste the code instead of the example below.

\begin{CCSXML}
<ccs2012>
<concept>
<concept_id>10011007.10011074.10011099.10011692</concept_id>
<concept_desc>Software and its engineering~Formal software verification</concept_desc>
<concept_significance>500</concept_significance>
</concept>
<concept>
<concept_id>10011007.10011074.10011099.10011102.10011103</concept_id>
<concept_desc>Software and its engineering~Software testing and debugging</concept_desc>
<concept_significance>300</concept_significance>
</concept>
</ccs2012>
\end{CCSXML}

\ccsdesc[500]{Software and its engineering~Formal software verification}
\ccsdesc[300]{Software and its engineering~Software testing and debugging}

%%
%% Keywords. The author(s) should pick words that accurately describe
%% the work being presented. Separate the keywords with commas.
\keywords{object-oriented languages, hash code, mutable classes, runtime verification}

\maketitle

\section{Introduction}

Most mainstream object-oriented languages provide a notion of equality between objects which can be customized to be weaker than
reference equality, and which is coupled with the customizable notion of object hash code \cite{Bloch18}. Such two notions are provided
through two corresponding methods defined in the predefined class \lstinline{Object} class which is a the root of the inheritance hierarchy; hence,
they are inherited or redefined in any class, and callable on any type of object.

For this reason, they are pervasive in object-oriented code and the correct functioning of some functionalities of many libraries rely on them;
hence, their incorrect redefinition or use may have a serious impact on software reliability and safety.

A classical example of useful redefinition of equality is for value classes, where typically a notion of logical equality is needed which differs
from reference equality \cite{Bloch18}. 
Obeying the general contract for equality is challenging, and equality redefinition invalidates the general contract for computing
object hash codes.

Indeed, implementations of hash tables typically use equality and object hash codes, therefore
they have to satisfy the general contract requiring that if two objects are equal,
then the same hash code must be computed for them. If this requirement is not satisfied, then hash tables fail to behave correctly.
Indeed, to find an element in a hash table, its hash code is computed to identify its bucket, then
equality is used to test whether the element is contained in such a bucket. If an equal element is already contained in the hash table, but in
a different bucket, because the computed hash code is different, then the element cannot be found.

While this problem is well known and there have been some attempts to detect it with verification techniques \cite{Bloch18,OkanoHSON19},
the similar problem of hash code redefinition for mutable classes has been overlooked.
Despite mutable classes should not redefine equality and hash code, many widely used API libraries do that in Java and other similar languages.
When objects of such classes are used as keys in hash tables, programs may exhibit unexpected and unpredictable behavior. Indeed,
if an object is modified while contained in a hash table, then most likely the same object can no longer be found in the table
even though no operations have been performed on the hash table.

Verifying that mutable objects with redefined hash code are used correctly in hash tables is not an easy task, because state modification needs to be tracked with a certain precision and rather complex control-oriented properties \cite{AhrendtCPS17} have to be ensured.

In this paper we present a solution based on Runtime Verification (RV), a dynamic verification technique where
the event trace generated by a single run of
the system under scrutiny (SUS) is checked by a monitor compiled from the formal specification which defines the correct behavior
of the SUS.
Events are usually emitted through code instrumentation of the SUS, to allow monitors to observe them;
although specification of properties and code instrumentation can be mixed together, decoupling the two activities favors abstraction, reuse and
interoperability of the generated monitors.

Monitors can be run \emph{offline} or \emph{online};
in offline RV monitors analyze event traces  after that they have been generated and collected into log files
by the instrumented SUS. In online RV traces are analyzed real-time to allow error detection to trigger specific actions on the SUS. 
Offline RV \cite{Colombo2022} is often exploited as a useful solution to integrate other approaches as debugging and testing;  %when errors are not critical or online RV would be too costly;
when the overhead of code instrumentation and of the executed monitors is limited, online RV can be employed to allow error recovery in critical scenarios. 

RV is complementary to formal verification and testing:
as formal verification, RV is based on a specification formalism; as happens for software testing, it
scales well to real systems and complex properties, but cannot guarantee exhaustiveness.
Differently from testing, it is particularly useful to ensure control-oriented properties 
and detect errors due to non-deterministic behavior \cite{havelund2004,sharma2009}.
Furthermore, online monitoring allows runtime contract enforcement, fault protection and automatic program repair.
Finally, several RV tools are based on abstract and intuitive specification languages that can be easily mastered by the user
and favor system agnosticism, portability, reuse, and interoperability.

Our proposed solution uses \rml\footnote{\href{https://rmlatdibris.github.io/}{https://rmlatdibris.github.io/}.}, a rewriting-based Domain Specific Language (DSL) for RV
which allows definition of formal specifications independently of code instrumentation and of the programming language used
to develop the software to be verified. 

The paper is structured as follows.
\Cref{sec:example} introduces the problem in detail and analyzes it in the context of several mainstream object-oriented languages,
\Cref{sec:rml} provides an introduction to \rml, \Cref{sec:spec} presents the proposed solution and discussed possible generalization,
\Cref{sec:rel-concl} is devoted to the related work and conclusions.

\section{Problems with hashCode and mutable classes}

While correctness issues concerned with the relationship between methods \lstinline{equals} and \lstinline{hashCode} are well-known
in Java and C\# \cite{Bloch18,OkanoHSON19}, less attention is devoted to the potentially dangerous effects of the redefinition of 
\lstinline{hashCode} in mutable classes when their instances are used in container objects implemented with hash tables.

In Java such a problem is more serious because all mutable classes implementing interface \lstinline{java.util.Collection} redefine
method \lstinline{hashCode} as their instances where immutable (i.e. value objects).

Let us consider this simple example:
\begin{lstlisting}[numbers=left]
var sset = new HashSet<Set<Integer>>();
var s = new HashSet<>(asList(1,2,3));
sset.add(s); // sset is {{1,2,3}}
assert sset.contains(s); // success
s.remove(1);
assert sset.contains(s); // failure
s.add(1);
assert sset.contains(s); // success
\end{lstlisting}
Two sets are created with class \lstinline{java.util.HashSet}; in such a class method \lstinline{hashCode} of the elements is used to identify the
bucket where they are stored in the hash table, and method \lstinline{equals} to search them in the bucket.
After execution of the first three lines \lstinline{sset} contains \lstinline{set} as stated by the succesfull assertion at line 4; in turn, \lstinline{set}
contains the three elements of type \lstinline{Integer} corresponding to 1, 2 and 3.

At line 5 element 1 is removed from \lstinline{set} and at the next line the same assertion is checked again; this time the assertion fails, although no method has been invoked on \lstinline{sset} and, hence, its state should be the same as in the previous assertion.


\begin{lstlisting}
var sset=new HashSet<ISet<int>>();
var set=new HashSet<int>(new int[]{1,2,3});
sset.Add(set);
Debug.Assert(sset.Contains(set));
set.Remove(1);
Debug.Assert(sset.Contains(set));
set.Add(1);
Debug.Assert(sset.Contains(set));
\end{lstlisting}


\begin{lstlisting}[language=Python]
sset=set()                                                           
s=set([1,2,3])                                                         
sset.add(s) # TypeError: unhashable type: 'set'       
\end{lstlisting}

\begin{lstlisting}[language=Javascript]
let sset=new Set();                                                           
let s=new Set([1,2,3]); // sset is {{1,2,3}}                                                         
sset.add(s);
console.assert(sset.has(s)); // success
s.delete(1);
console.assert( sset.has(s)); // success
s.add(1);
console.assert( sset.has(s)); // success
\end{lstlisting}





\section{\rml}
\label{sec:rml}
\lstset{morekeywords={matches,not,let}}
\rml\cite{RML2021} is a rewriting-based DSL for RV which allows developers to define formal specifications independently of code instrumentation.

It  is based on the notion of \emph{event type} (denoting a set of events) and \emph{trace expression} (denoting a set of event traces),
and it is implemented by a compiler, which generates monitors able to run independently of the SUS and of its instrumentation. 
%The language is based on previous work on RV and global types \cite{CastagnaEtAl12,AnconaBB0CDGGGH16,AnconaFM17},
%applied to several contexts, including verification of interaction protocols in multi-agent systems \cite{AnconaDM12,AnconaBFMT14, BriolaMA14}.

\paragraph{Events}
In \rml an \emph{event} is any observation relevant for monitoring the SUS.
Events are represented in a general way with object literals and
consist of properties which identify the type of event and the data associated with it. For instance,

\begin{lstlisting}
{event:"func_post",targetId:9,name:"add",
    res:true,args:[1]}
\end{lstlisting}          
represents the event
`call to method \lstinline{add} on target object with id 9 and with argument 1  has returned value \lstinline{true}.
Another type of events which are often useful to monitor is \lstinline{'func_pre'}, that is, entering a constructor or method call; of course, in this
case, no information on the returned value can be provided. Depending on the features of the instrumentation tool, other finer grained types,
as reading or updating a field, can be used, but at the cost of making specifications more coupled with the specific application that needs to be
verified, and, hence, less reusable and portable.

An \rml specification defines the set of event traces expected from correct runs of the SUS; the monitor automatically generated from
such a specification checks that the trace generated by a single run of the SUS belongs to such a set.

\paragraph{Event Types}
The basic blocks which constitute an \rml specification are \emph{patterns} built from
\emph{event types} defining sets of events.

Event types are defined with clauses:
\begin{lstlisting}[basicstyle=\ttfamily\scriptsize]
add(hash_id,elem_id) matches {event:'func_post', targetId:hash_id, name:'add', argIds:[elem_id], res:true};
\end{lstlisting}
In this example \lstinline{add} matches events parametric in the
ids \lstinline{hash_id} and \lstinline{elem_id} of the target and argument of the call. While property
\lstinline{args} is useful when arguments are primitive values, \lstinline{argIds} is used when arguments are objects, denoted by their unique id;
similarly, for the returned value the two properties \lstinline{res} and \lstinline{resultId} are available.

%% As a more general example
%% \begin{lstlisting}[basicstyle=\ttfamily\scriptsize]
%% modify(id) matches {event:'func_post', targetId:id, name:'add' | 'addAll' | 'remove' | 'removeAll' | 'removeIf' | 'retainAll', argIds:[_], res:true};
%% \end{lstlisting}
%% defines the event type matching all calls on target with object identifier id which modify it. 

%\noindent then $\mtch(\ev,\mathit{open}(x))$ successfully returns the substitution
%$\{ x \mapsto 42 \}$.

%%Trace expressions have been extended with variables in \cite{AnconaFM17}.

\rml allows also the definition of event types derived from others:

\begin{lstlisting}[basicstyle=\ttfamily\scriptsize]
not_add(hash_id) not matches add(hash_id,_);
op(hash_id,elem_id) matches {targetId:hash_id} | {targetId:elem_id};
\end{lstlisting}  
The event pattern \lstinline{not_add(hash_id)} matches all events which do not correspond to the return from
method \lstinline{add} called on target \lstinline{hash_id}; the wildcard \lstinline!_! is used when a value is not relevant for
the definition of the event type.

The event pattern \lstinline{op(hash_id,elem_id)} matches all events matching either \lstinline!{targetId:hash_id}! or
\lstinline!{targetId:elem_id}!, that is, all calls on target \lstinline{hash_id} or \lstinline{elem_id}.

\paragraph{Trace Expressions}
The basic layer of \rml are expressions that define sets of event traces and built by combining together event patterns with primitive and derived operators.
The former kind of operators includes, among others, the following binary operators on sets of event traces:
\begin{itemize}
	%% \item $\emptyseq$ (\emph{empty trace}): the singleton set $\{\emptyseq\}$ containing  the empty event trace $\emptyseq$;
	%% \item $\eventTy\prefixop\tau$ (\emph{prefix}): the set of all traces whose first event $\ev$ matches the event type $\eventTy$, and the remaining part is a trace of $\tau$;
	\item \emph{concatenation} (denoted by juxtaposition);
	\item \emph{intersection} \lstinline{/\};
	\item \emph{union} \lstinline{\/}; 
	\item \emph{shuffle} \lstinline{|}. 
	%% \item $\var{x}{\tau}$ (\emph{binder}): it binds the free occurrences of $\xv$ in $\tau$;
	%% \item $\eventTy\filterop\tau$ (\emph{filter}):
	%% denoting the set of all traces contained in $\tau$, when they are deprived af all events that do not match $\eventTy$ (theoretically, this operator can also be derived from the others).
\end{itemize}
Other useful derivable operators are available, including the standard postfix operators
\lstinline{?}, \lstinline{+} and \lstinline{*}, borrowed from regular expressions,  the constant \lstinline{all}, which denotes the universe of all traces,
and the  conditional filter operator \lstinline{_ >> _ : _}.
%% which denotes the set of all
%% traces verifying $\te_1$ when only the events matching $\eventTy$ are kept, and
%% $\te_2$ when the events matching $\eventTy$ are filtered out.

%Trace expressions are regular terms (a.k.a. cyclic) \cite{Courcelle83}, thus there is no need for an explicit recursion operator.
The formal semantics of trace expressions is defined in terms of a labeled transition system \cite{RML2021}.

As a very simple example, the specification \lstinline{Main} in \Cref{example} defines the set of event traces starting with a call to
a constructor of class \lstinline{HashSet} returning the object id 42, followed by zero or more calls to method \lstinline{add} on target id 42 with returned value \lstinline{true} and ending with a call to method \lstinline{remove} on the same target id with returned value \lstinline{true}.
%% The fact that the specification ends with \lstinline{all} means that after the event matching \lstinline{remove(42)} any event trace is allowed, that is, no further verification is required.

\begin{figure}[h]
\begin{lstlisting}[basicstyle=\ttfamily\scriptsize]
new_hash(hash_id) matches {event:'func_post', name:'HashSet', resultId:hash_id};
remove(hash_id) matches {event:'func_post', targetId:hash_id, name:'remove', res:true};
add(hash_id) matches {event:'func_post', targetId:hash_id, name:'add', res:true};

Main = new_hash(42) add(42)* remove(42);
\end{lstlisting}
\caption{Example of specification.}\label{example}
\end{figure}

\paragraph{Parametric Specifications}
The expression \lstinline{new_hash(42) add(42)* remove(42)} is of very limited use because it refers to a specific object id;
however, \rml provides a \lstinline{let} construct \cite{AnconaFM17} for declaring existentially quantified variables.
With such an abstraction and the shuffle operator, it is possible to write a \emph{parametric} specification working for any instance of class \lstinline{HashSet}:
\begin{lstlisting}[basicstyle=\ttfamily\scriptsize]
Main = {let hash_id; new_hash(hash_id) (add(hash_id)* remove(hash_id) | Main)};
\end{lstlisting}
When the first event matches \lstinline{new_hash(hash_id)}, \lstinline{hash_id} is bound to the specific id returned by the constructor in the two
occurences on the left-hand-side of the shuffle. The binding does not affect the recursive use of \lstinline{Main}
because its nested \lstinline{let} declaration masks the outer one, thus allowing to properly verify the specified property
for any new instance of the class.

It is worth noting that such a specification pattern where recursion occurs on one side of shuffle (or intersection, as shown in the next section)
is quite useful for specifying several kinds of properties \cite{RML2021} which cannot be specified with regular expressions (and hence
with LTL which is less expressive \cite{Strejcek2004}). Indeed, while regular expressions are closed w.r.t. shuffle, they are not
w.r.t. iterated shuffle \cite{FlickK12}; intersection allows even more expressive power since context-free languages are not closed w.r.t. such an operation \cite{RML2021}.

%% At this point the shuffle operator is crucial\footnote{In order for the shuffle to work, we assume that every \(\opent\) operation always gives a fresh file descriptor, which can be reasonably assumed to be ensured by the operating system.}: following events are allowed to belong either to \(\tau\) (operations on new files, in the correct order) or to \(\tau'\{\avar{fd}\mapsto x\}\) (note the substitution!) were further operations on \(x\) will be checked.

\paragraph{Generic Specifications}
\rml provides a further abstract layer with \emph{generic} specifications,
to enhance modularity and reuse and increase its expressive power \cite{RML2021}.
%%offer more opportunities for reuse and increase the expressive power of \rml with the conditional expression.
With the generic \lstinline{Spec<hash_id>}, the specification above
can be generalized as follows to make it more readable and possibly reusable:
\begin{lstlisting}[basicstyle=\ttfamily\scriptsize]
Spec<hash_id> =  add(hash_id)* remove(hash_id); 
Main = {let hash_id; new_hash(hash_id) ( Spec<hash_id> | Main)};
\end{lstlisting}
%% To modify the specification of the allowed traces after the creation of an \lstinline{HashSet} instance, only the definition
%% \lstinline{File<fdesc>} needs to be redefined:
%% \begin{lstlisting}[basicstyle=\ttfamily\scriptsize]
%% ...
%% chmod(fd) matches {event:'func_pre', name:'fs.fchmodSync', args:[fd,_]};

%% File<fdesc> = chmod(fdesc)? write(fdesc)* close(fdesc); // chmod allowed but only before writing
%% Main = {let fdesc; open(fdesc) (File<fdesc>  | Main)}; // Main specification unchanged
%% \end{lstlisting}
%% Recursion together with the conditional expression are used to define more expressive generic specifications:
%% \begin{lstlisting}[basicstyle=\ttfamily\scriptsize]
%% Repeat<n> = if (n>0) eventType Repeat<n-1> else empty;
%% \end{lstlisting}
%% The generic specification \lstinline{Repeat<$n$>} defines all traces of length $n$ where all events match
%% the event type pattern \lstinline{eventType}.

\section{A specification of safe use of collections in hash sets}
In this section we show how it is possible to define a specification in \rml for dynamically verifying that hash sets and their elements of
type \lstinline{Collection} are managed correctly to avoid the issue highlighted by the examples in the previous section.

The only methods of \lstinline{Collection<E>} that can modify the state of a collection are \lstinline{add(E)} and \lstinline{remove(E)}; other methods, as \lstinline{addAll} and \lstinline{removeAll}, are defined in terms of the primitive ones \lstinline{add} and \lstinline{remove}, hence the specification we consider here covers also them. However, there are additional methods contained in subtypes of collection, consider for instance method
\lstinline{add(int,E)} and \lstinline{remove(int)} of \lstinline{List}, which cannot be monitored through \lstinline{add(E)} and \lstinline{remove(E)}. Possible generalization of the solution presented here are discussed in the last part of this section.

\paragraph{Events and event types} An interesting feature of \lstinline{add} and \lstinline{remove} is that they both return true if and only if the operation modifies the collection, hence modifications can be easily monitored at runtime, and it is possible to write a specification based only on events of type \lstinline{'func_post'}.

\begin{lstlisting}[basicstyle=\ttfamily\scriptsize]
add(hash_id,elem_id) matches {event:'func_post', targetId:hash_id, name:'add', argIds:[elem_id], res:true};
remove(hash_id,elem_id) matches {event:'func_post', targetId:hash_id, name:'remove', argIds:[elem_id], res:true};
\end{lstlisting}

After an event matches \lstinline{add(hash_id,elem_id)} the specification needs to verify that 
element \lstinline{elem_id}, which has just been inserted in the hash set \lstinline{hash_id}, is not modified 
until the element is removed from the set, that is, an event matching \lstinline{remove(hash_id,elem_id)} occurs.


\begin{figure}[h]
\begin{lstlisting}[basicstyle=\ttfamily\scriptsize]
new_hash(hash_id) matches {event:'func_post', name:'HashSet', resultId:hash_id};
not_new_hash not matches new_hash(_);
modify(targ_id) matches {event:'func_post', targetId:targ_id, name:'add' | 'remove', res:true};
not_modify_remove(hash_id,elem_id) not matches remove(hash_id,elem_id) | modify(elem_id);
add(hash_id,elem_id) matches {event:'func_post', targetId:hash_id, name:'add', argIds:[elem_id], res:true};
not_add(hash_id) not matches add(hash_id,_);
remove(hash_id,elem_id) matches {event:'func_post', targetId:hash_id, name:'remove', argIds:[elem_id], res:true};
op(hash_id,elem_id) matches {targetId:hash_id} | {targetId:elem_id};
relevant matches new_hash(_) | modify(_);

Main = not_new_hash* {let hash_id;new_hash(hash_id) (SafeHashTable<hash_id> /\ Main)}?;
SafeHashTable<hash_id> = not_add(hash_id)* {let elem_id;add(hash_id,elem_id) (SafeHashElem<hash_id,elem_id> /\ SafeHashTable<hash_id>)}?;
SafeHashElem<hash_id,elem_id> = op(hash_id,elem_id) >> not_modify_remove(hash_id,elem_id)* remove(hash_id,elem_id) all;
\end{lstlisting}
\caption{Specification of safe hash setd.}\label{list:hash}
\end{figure}

\section{Related Works and Conclusions}
\label{sec:rel-concl}

The need to understand the behaviour of a Java piece of code is as old as Java itself. 

Early attempts to visualize Java programs date back to the end of the millennium. They were initially motivated by teaching reasons \cite{dershem1999java}, and became soon a fundamental engineering step for developing correct and safe Java applications \cite{DBLP:conf/wcre/Systa00,DBLP:conf/dagstuhl/PauwJMSVY01}. The Java visualization research strand is still active \cite{DBLP:journals/spe/JayaramanJL17,DBLP:journals/spe/PJJS21} but since -- in order to visualize a program behavior -- it is necessary to trace it \cite{DBLP:conf/dagstuhl/Mehner01}, most efforts are currently oriented towards the more general problem of Java tracing.
% and, being a requirement for tracing, of Java instrumentation. 

Different approaches to tracing exist, mainly depending on which part of the Java architecture, the bytecode, the source code, the JVM, is modified or instrumented to make the tracing possible.

In a work dating back 2001 \cite{DBLP:journals/fgcs/BechiniP01}, Bechini and Prete present a solution for tracing and replaying Java concurrent applications based on the automatic instrumentation of the original source code.

A less invasive approach is MuTT \cite{DBLP:conf/ACISicis/LiuX09} that works on top of JPDA (the Java Platform Debugger Architecture, available for old JDKs) and exploits JPDA features to collect the run-time information of multi-threaded Java programs without source code or JVM instrumentation.

More recently, JBInsTrace \cite{DBLP:journals/scp/CasertaZ14} computes complex dynamic metrics used to categorize programs according to dynamic metrics related to program size and structure, use of data structures, use of
polymorphism, memory footprint and concurrency. 
To this aim, JBInsTrace instruments and traces Java bytecode. It does not alter the JVM and does not statically modify class files. 
%Via tracing, static information about source code is produced, as well as a  fine grained trace of Java software execution, that allow detailed analysis of the runtime. 

When tracing takes place while the program is running, the effect of tracing is indeed a runtime monitoring of the program's behavior or, using the terminology adopted in this paper, its runtime verification.

Indeed, runtime verification of Java programs started to be addressed in 2001, when the Java PathExplorer was developed \cite{havelund2001java}. Java PathExplorer tested the execution traces of the Java program against high level specifications expressed as temporal logic formulae. An initial prototype of the tool was applied to the executive module of the planetary Rover K9, developed at NASA Ames. 

JASSDA \cite{BrorkensM02} was developed one year after Java PathExplorer. It is a RV framework for Java programs
based on CSP-like specifications and implemented in Java. JASSDA is very simple and does not support concatenation; parametricity
is obtained through slicing. 

PQL \cite{MartinLL05}  is an expressive language supporting RV of open-source Java
applications that allows specifications of properties covering the closure of context-free languages combined with intersection;
however, it does not support shuffle, and parametricity.  Its implementation is based on Java, Python and DataLog. 


LARVA \cite{ColomboPS09} is a RV tool expressly designed for checking real-time properties of Java programs.
Properties are specified in DATEs \cite{DATEs}
based on an extension of timed automata; in particular, it supports
symbolic states to guard transitions, replication of automata, and 
CCS-like communication between automata. LARVA is implemented in Java and code instrumentation is based on AspectJ.

SAGA  \cite{BoerGouw14} is another framework for RV of Java programs
based on attribute grammars. With attribute grammars it is possible to support parametricity and to mix specifications
with code instrumentation by exploring the full computational power of Java. Its implementation exploits Java, ANTLR and Rascal.

Table \ref{comparison} provides a summary of the comparison of RML with some of the most widely used tools for RV of Java programs, thas shows that 
RML is SUS-agnostic verification with a strong separation between instrumentation and specification of the properties to be verified. It supports parametricity, generics, and it allows to express non context-free properties. 

\begin{table*}
    \includegraphics[width=0.6\textwidth,keepaspectratio=true]{tabella-viv3.jpg}[htb!]
\caption{Tools for RV of Java programs and their comparison with RML.}
\label{comparison}
\end{table*}


\bibliographystyle{ACM-Reference-Format}

\balance

\bibliography{biblio}

%%\appendix
\label{appendix}
\section{Example of verified Java code}
\begin{lstlisting}[numbers=left]
var setset = new HashSet<Set<Integer>>();
var set1 = new HashSet<Integer>();
var set2 = new HashSet<Integer>();
set1.add(1);
set2.add(2);
setset.add(set1);
set1.contains(1);
setset.add(set2);
setset.remove(set1);
set1.remove(1);
set2.remove(1);
//s2.remove(2);
setset.remove(set2);
set1.add(1);
set2.add(2);
\end{lstlisting}

\end{document}
\endinput
%%
%% End of file `sample-sigconf.tex'.
