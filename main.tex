%%
%% This is file `sample-sigconf.tex',
%% generated with the docstrip utility.
%%
%% The original source files were:
%%
%% samples.dtx  (with options: `sigconf')
%% 
%% IMPORTANT NOTICE:
%% 
%% For the copyright see the source file.
%% 
%% Any modified versions of this file must be renamed
%% with new filenames distinct from sample-sigconf.tex.
%% 
%% For distribution of the original source see the terms
%% for copying and modification in the file samples.dtx.
%% 
%% This generated file may be distributed as long as the
%% original source files, as listed above, are part of the
%% same distribution. (The sources need not necessarily be
%% in the same archive or directory.)
%%
%%
%% Commands for TeXCount
%TC:macro \cite [option:text,text]
%TC:macro \citep [option:text,text]
%TC:macro \citet [option:text,text]
%TC:envir table 0 1
%TC:envir table* 0 1
%TC:envir tabular [ignore] word
%TC:envir displaymath 0 word
%TC:envir math 0 word
%TC:envir comment 0 0
%%
%%
%% The first command in your LaTeX source must be the \documentclass
%% command.
%%
%% For submission and review of your manuscript please change the
%% command to \documentclass[manuscript, screen, review]{acmart}.
%%
%% When submitting camera ready or to TAPS, please change the command
%% to \documentclass[sigconf]{acmart} or whichever template is required
%% for your publication.
%%
%%
\documentclass[sigconf,review]{acmart}

%%
%% \BibTeX command to typeset BibTeX logo in the docs
\AtBeginDocument{%
  \providecommand\BibTeX{{%
    Bib\TeX}}}

%% Rights management information.  This information is sent to you
%% when you complete the rights form.  These commands have SAMPLE
%% values in them; it is your responsibility as an author to replace
%% the commands and values with those provided to you when you
%% complete the rights form.
\setcopyright{acmcopyright}
\copyrightyear{2023}
\acmYear{2023}
\acmDOI{XXXXXXX.XXXXXXX}

%% These commands are for a PROCEEDINGS abstract or paper.
\acmConference[Conference acronym 'XX]{Make sure to enter the correct
  conference title from your rights confirmation emai}{July,
  2023}{Seattle, WA}
%%
%%  Uncomment \acmBooktitle if the title of the proceedings is different
%%  from ``Proceedings of ...''!
%%
%%\acmBooktitle{Woodstock '18: ACM Symposium on Neural Gaze Detection,
%%  June 03--05, 2018, Woodstock, NY}
\acmPrice{}
\acmISBN{}


%%
%% Submission ID.
%% Use this when submitting an article to a sponsored event. You'll
%% receive a unique submission ID from the organizers
%% of the event, and this ID should be used as the parameter to this command.
%%\acmSubmissionID{123-A56-BU3}

%%
%% For managing citations, it is recommended to use bibliography
%% files in BibTeX format.
%%
%% You can then either use BibTeX with the ACM-Reference-Format style,
%% or BibLaTeX with the acmnumeric or acmauthoryear sytles, that include
%% support for advanced citation of software artefact from the
%% biblatex-software package, also separately available on CTAN.
%%
%% Look at the sample-*-biblatex.tex files for templates showcasing
%% the biblatex styles.
%%

%%
%% The majority of ACM publications use numbered citations and
%% references.  The command \citestyle{authoryear} switches to the
%% "author year" style.
%%
%% If you are preparing content for an event
%% sponsored by ACM SIGGRAPH, you must use the "author year" style of
%% citations and references.
%% Uncommenting
%% the next command will enable that style.
%%\citestyle{acmauthoryear}

%%
%% end of the preamble, start of the body of the document source.
\begin{document}

%%
%% The "title" command has an optional parameter,
%% allowing the author to define a "short title" to be used in page headers.
\title{Runtime verification of use of hash code of mutable objects}

%%
%% The "author" command and its associated commands are used to define
%% the authors and their affiliations.
%% Of note is the shared affiliation of the first two authors, and the
%% "authornote" and "authornotemark" commands
%% used to denote shared contribution to the research.
\author{Davide Ancona}
\email{davide.ancona@unige.it}
\orcid{0000-0002-6297-201}
\author{Angelo Ferrando}
\email{angelo.ferrando@unige.it}
\orcid{0000-0002-8711-4670}
\author{Viviana Mascardi}
\email{viviana.mascardi@unige.it}
\orcid{0000-0002-2261-9926}

\affiliation{%
  \institution{DIBRIS, Universit\`a di Genova}
%  \streetaddress{P.O. Box 1212}
 % \city{Dublin}
 % \state{Ohio}
  \country{Italy}
 % \postcode{43017-6221}
}

%%
%% By default, the full list of authors will be used in the page
%% headers. Often, this list is too long, and will overlap
%% other information printed in the page headers. This command allows
%% the author to define a more concise list
%% of authors' names for this purpose.
\renewcommand{\shortauthors}{Ancona et al.}

%%
%% The abstract is a short summary of the work to be presented in the
%% article.
\begin{abstract}

\end{abstract}

%%
%% The code below is generated by the tool at http://dl.acm.org/ccs.cfm.
%% Please copy and paste the code instead of the example below.
%%
%% \begin{CCSXML}
%% <ccs2012>
%%  <concept>
%%   <concept_id>10010520.10010553.10010562</concept_id>
%%   <concept_desc>Computer systems organization~Embedded systems</concept_desc>
%%   <concept_significance>500</concept_significance>
%%  </concept>
%%  <concept>
%%   <concept_id>10010520.10010575.10010755</concept_id>
%%   <concept_desc>Computer systems organization~Redundancy</concept_desc>
%%   <concept_significance>300</concept_significance>
%%  </concept>
%%  <concept>
%%   <concept_id>10010520.10010553.10010554</concept_id>
%%   <concept_desc>Computer systems organization~Robotics</concept_desc>
%%   <concept_significance>100</concept_significance>
%%  </concept>
%%  <concept>
%%   <concept_id>10003033.10003083.10003095</concept_id>
%%   <concept_desc>Networks~Network reliability</concept_desc>
%%   <concept_significance>100</concept_significance>
%%  </concept>
%% </ccs2012>
%% \end{CCSXML}

%% \ccsdesc[500]{Computer systems organization~Embedded systems}
%% \ccsdesc[300]{Computer systems organization~Redundancy}
%% \ccsdesc{Computer systems organization~Robotics}
%% \ccsdesc[100]{Networks~Network reliability}

%%
%% Keywords. The author(s) should pick words that accurately describe
%% the work being presented. Separate the keywords with commas.
%% \begin{CCSXML}
%% <ccs2012>
%%  <concept>
%%   <concept_id>10010520.10010553.10010562</concept_id>
%%   <concept_desc>Computer systems organization~Embedded systems</concept_desc>
%%   <concept_significance>500</concept_significance>
%%  </concept>
%%  <concept>
%%   <concept_id>10010520.10010575.10010755</concept_id>
%%   <concept_desc>Computer systems organization~Redundancy</concept_desc>
%%   <concept_significance>300</concept_significance>
%%  </concept>
%%  <concept>
%%   <concept_id>10010520.10010553.10010554</concept_id>
%%   <concept_desc>Computer systems organization~Robotics</concept_desc>
%%   <concept_significance>100</concept_significance>
%%  </concept>
%%  <concept>
%%   <concept_id>10003033.10003083.10003095</concept_id>
%%   <concept_desc>Networks~Network reliability</concept_desc>
%%   <concept_significance>100</concept_significance>
%%  </concept>
%% </ccs2012>
%% \end{CCSXML}

%% \ccsdesc[500]{Computer systems organization~Embedded systems}
%% \ccsdesc[300]{Computer systems organization~Redundancy}
%% \ccsdesc{Computer systems organization~Robotics}
%% \ccsdesc[100]{Networks~Network reliability}

%%
%% Keywords. The author(s) should pick words that accurately describe
%% the work being presented. Separate the keywords with commas.
\keywords{Java, hash code, mutable objects, runtime verification}

\maketitle

\section{Introduction}

Most mainstream object-oriented languages provide a notion of equality between objects which can be customized to be weaker than
reference equality, and which is coupled with the customizable notion of object hash code \cite{Bloch18}. Such two notions are provided
through two corresponding methods defined in the predefined class \lstinline{Object} which is at the root of the inheritance hierarchy; hence,
they are inherited or can be redefined in any class, and are callable on any type of object.

For this reason, they are pervasive in object-oriented code and the correct functioning of some features in many libraries rely on them;
hence, their incorrect redefinition or use may have a serious impact on software reliability and safety.

A classical example of useful redefinition of equality is for value classes, where typically a notion of logical equality is needed which differs
from reference equality. 
Obeying the general contract for equality is challenging, and equality redefinition invalidates the general contract for computing
object hash codes \cite{Bloch18}.

Indeed, implementations of hash tables typically use equality and object hash codes, therefore
a general contract has to be satisfied: if two objects are equal,
then the same hash code must be computed for them.

If this requirement is not satisfied, then hash tables fail to behave correctly.
Indeed, to find an element in a hash table, its hash code is computed to identify its bucket, then
equality is used to test whether the element is contained in such a bucket. If an equal element is already contained in the hash table, but in
a different bucket, because the computed hash code is different, then the element cannot be found.

While this problem is well known and there have been some attempts to detect it with verification techniques \cite{Bloch18,OkanoHSON19},
hash code redefinition for mutable classes has been overlooked.
When objects of such classes are used as keys in hash tables, programs may exhibit unexpected and unpredictable behavior. Indeed,
if an object is modified while contained in a hash table, then most likely the same object can no longer be found in the table
even though no operations have been performed on the hash table.

Redefinition of equality and hash code in mutable classes is unsafe, as pointed out in
the documentation for \lstinline{java.util.Set} \cite{NelsonEtAl2010} and, similarly, \lstinline{java.util.Map}: 
``\emph{Great care must be exercised if mutable objects are used as set elements. The behavior of a set is not specified if the value of an object is changed in a manner that affects equals comparisons while the object is an element in the set. A special case of this prohibition is that it is not permissible for a set to contain itself as an element.}''

Despite this note, many widely used API libraries do that in Java and other similar languages. 
Verifying that mutable objects with redefined hash code are used correctly in hash tables is not an easy task, because state modification needs to be tracked with a certain precision and rather complex control-oriented properties \cite{AhrendtCPS17,AnconaDF18} have to be ensured.

In this paper we present a solution based on Runtime Verification (RV), a dynamic verification technique where
a single execution of the system under scrutiny (SUS) is abstracted by an event trace
which is checked by a monitor compiled from the formal specification defining the correct behavior
of the SUS.

Events are usually generated by instrumented code of the SUS, and logged or directly sent to the monitor.
Although specification of properties and code instrumentation can be mixed together, decoupling the two activities favors abstraction, reuse and
interoperability of the generated monitors.

Monitors can be \emph{offline} or \emph{online};
in offline RV a trace is typically generated by the instrumented SUS and stored into a log file and then is analyzed by the monitor.
In online RV traces are analyzed real-time to allow error detection to trigger specific actions on the SUS. 
Offline RV \cite{Colombo2022} is a useful solution to integrate other approaches as debugging and testing;  %when errors are not critical or online RV would be too costly;
online RV can be employed to allow error recovery in critical scenarios, providing that such a choice is compatible with the overhead of code instrumentation and of the monitor execution. 

RV is complementary to formal verification and testing:
as formal verification, RV is based on a specification formalism; as happens for software testing, it
scales well to real systems and complex properties, but cannot guarantee exhaustiveness.
Differently from testing, it is particularly useful to ensure control-oriented properties \cite{AhrendtCPS17,AnconaDF18} 
and detect errors due to non-deterministic behavior \cite{havelund2004,sharma2009}.
Furthermore, online monitoring allows runtime contract enforcement, fault protection and automatic program repair.
Finally, several RV tools are based on abstract and intuitive specification languages that can be easily mastered by the user
and favor system agnosticism, portability, reuse, and interoperability.

Our proposed solution is based on offline RV and uses  \rml\footnote{\href{https://rmlatdibris.github.io/}{https://rmlatdibris.github.io/}.}, a rewriting-based Domain Specific Language (DSL) for RV
which allows definition of formal specifications independently of code instrumentation and of the programming language used
to develop the software to be verified. The choice of \rml makes our solution easily portable to different Java-like languages.
Offline RV has been preferred over online RV because the main aim is to detect unsafe use of hash tables; this allows also a simpler solution
which minimize overhead.

The paper is structured as follows.
\Cref{sec:example} introduces the problem in detail and analyzes it in the context of several mainstream object-oriented languages,
\Cref{sec:rml} provides an introduction to \rml, \Cref{sec:spec} presents the proposed solution and discussed possible generalization,
\Cref{sec:rel-concl} is devoted to the related work and conclusions.


\bibliographystyle{ACM-Reference-Format}
\bibliography{biblio}

\end{document}
\endinput
%%
%% End of file `sample-sigconf.tex'.
