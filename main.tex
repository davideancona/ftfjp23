%%
%% This is file `sample-sigconf.tex',
%% generated with the docstrip utility.
%%
%% The original source files were:
%%
%% samples.dtx  (with options: `sigconf')
%% 
%% IMPORTANT NOTICE:
%% 
%% For the copyright see the source file.
%% 
%% Any modified versions of this file must be renamed
%% with new filenames distinct from sample-sigconf.tex.
%% 
%% For distribution of the original source see the terms
%% for copying and modification in the file samples.dtx.
%% 
%% This generated file may be distributed as long as the
%% original source files, as listed above, are part of the
%% same distribution. (The sources need not necessarily be
%% in the same archive or directory.)
%%
%%
%% Commands for TeXCount
%TC:macro \cite [option:text,text]
%TC:macro \citep [option:text,text]
%TC:macro \citet [option:text,text]
%TC:envir table 0 1
%TC:envir table* 0 1
%TC:envir tabular [ignore] word
%TC:envir displaymath 0 word
%TC:envir math 0 word
%TC:envir comment 0 0
%%
%%
%% The first command in your LaTeX source must be the \documentclass
%% command.
%%
%% For submission and review of your manuscript please change the
%% command to \documentclass[manuscript, screen, review]{acmart}.
%%
%% When submitting camera ready or to TAPS, please change the command
%% to \documentclass[sigconf]{acmart} or whichever template is required
%% for your publication.
%%
%%
\documentclass[sigconf,review]{acmart}
\usepackage{listings}
\usepackage{js}
\usepackage{js}
\usepackage{xspace}

\newcommand{\rml}{RML\xspace}

\lstset{language=Java,basicstyle=\ttfamily\footnotesize,commentstyle=\itshape,morekeywords={assert},keywordstyle=\ttfamily\bfseries}
%%
%% \BibTeX command to typeset BibTeX logo in the docs
\AtBeginDocument{%
  \providecommand\BibTeX{{%
    Bib\TeX}}}

%% Rights management information.  This information is sent to you
%% when you complete the rights form.  These commands have SAMPLE
%% values in them; it is your responsibility as an author to replace
%% the commands and values with those provided to you when you
%% complete the rights form.
\setcopyright{acmcopyright}
\copyrightyear{2023}
\acmYear{2023}
\acmDOI{XXXXXXX.XXXXXXX}

%% These commands are for a PROCEEDINGS abstract or paper.
\acmConference[Conference acronym 'XX]{Make sure to enter the correct
  conference title from your rights confirmation emai}{July,
  2023}{Seattle, WA}
%%
%%  Uncomment \acmBooktitle if the title of the proceedings is different
%%  from ``Proceedings of ...''!
%%
%%\acmBooktitle{Woodstock '18: ACM Symposium on Neural Gaze Detection,
%%  June 03--05, 2018, Woodstock, NY}
\acmPrice{}
\acmISBN{}


%%
%% Submission ID.
%% Use this when submitting an article to a sponsored event. You'll
%% receive a unique submission ID from the organizers
%% of the event, and this ID should be used as the parameter to this command.
%%\acmSubmissionID{123-A56-BU3}

%%
%% For managing citations, it is recommended to use bibliography
%% files in BibTeX format.
%%
%% You can then either use BibTeX with the ACM-Reference-Format style,
%% or BibLaTeX with the acmnumeric or acmauthoryear sytles, that include
%% support for advanced citation of software artefact from the
%% biblatex-software package, also separately available on CTAN.
%%
%% Look at the sample-*-biblatex.tex files for templates showcasing
%% the biblatex styles.
%%

%%
%% The majority of ACM publications use numbered citations and
%% references.  The command \citestyle{authoryear} switches to the
%% "author year" style.
%%
%% If you are preparing content for an event
%% sponsored by ACM SIGGRAPH, you must use the "author year" style of
%% citations and references.
%% Uncommenting
%% the next command will enable that style.
%%\citestyle{acmauthoryear}

%%
%% end of the preamble, start of the body of the document source.
\begin{document}

%%
%% The "title" command has an optional parameter,
%% allowing the author to define a "short title" to be used in page headers.
\title{Runtime verification of hashCode in mutable classes}

%%
%% The "author" command and its associated commands are used to define
%% the authors and their affiliations.
%% Of note is the shared affiliation of the first two authors, and the
%% "authornote" and "authornotemark" commands
%% used to denote shared contribution to the research.
\author{Davide Ancona}
\email{davide.ancona@unige.it}
\orcid{0000-0002-6297-201}
\author{Angelo Ferrando}
\email{angelo.ferrando@unige.it}
\orcid{0000-0002-8711-4670}
\author{Viviana Mascardi}
\email{viviana.mascardi@unige.it}
\orcid{0000-0002-2261-9926}

\affiliation{%
  \institution{DIBRIS, Universit\`a di Genova}
%  \streetaddress{P.O. Box 1212}
 % \city{Dublin}
 % \state{Ohio}
  \country{Italy}
 % \postcode{43017-6221}
}

%%
%% By default, the full list of authors will be used in the page
%% headers. Often, this list is too long, and will overlap
%% other information printed in the page headers. This command allows
%% the author to define a more concise list
%% of authors' names for this purpose.
\renewcommand{\shortauthors}{Ancona et al.}

%%
%% The abstract is a short summary of the work to be presented in the
%% article.
\begin{abstract}

\end{abstract}

%%
%% The code below is generated by the tool at http://dl.acm.org/ccs.cfm.
%% Please copy and paste the code instead of the example below.
%%
%% \begin{CCSXML}
%% <ccs2012>
%%  <concept>
%%   <concept_id>10010520.10010553.10010562</concept_id>
%%   <concept_desc>Computer systems organization~Embedded systems</concept_desc>
%%   <concept_significance>500</concept_significance>
%%  </concept>
%%  <concept>
%%   <concept_id>10010520.10010575.10010755</concept_id>
%%   <concept_desc>Computer systems organization~Redundancy</concept_desc>
%%   <concept_significance>300</concept_significance>
%%  </concept>
%%  <concept>
%%   <concept_id>10010520.10010553.10010554</concept_id>
%%   <concept_desc>Computer systems organization~Robotics</concept_desc>
%%   <concept_significance>100</concept_significance>
%%  </concept>
%%  <concept>
%%   <concept_id>10003033.10003083.10003095</concept_id>
%%   <concept_desc>Networks~Network reliability</concept_desc>
%%   <concept_significance>100</concept_significance>
%%  </concept>
%% </ccs2012>
%% \end{CCSXML}

%% \ccsdesc[500]{Computer systems organization~Embedded systems}
%% \ccsdesc[300]{Computer systems organization~Redundancy}
%% \ccsdesc{Computer systems organization~Robotics}
%% \ccsdesc[100]{Networks~Network reliability}

%%
%% Keywords. The author(s) should pick words that accurately describe
%% the work being presented. Separate the keywords with commas.
%% \begin{CCSXML}
%% <ccs2012>
%%  <concept>
%%   <concept_id>10010520.10010553.10010562</concept_id>
%%   <concept_desc>Computer systems organization~Embedded systems</concept_desc>
%%   <concept_significance>500</concept_significance>
%%  </concept>
%%  <concept>
%%   <concept_id>10010520.10010575.10010755</concept_id>
%%   <concept_desc>Computer systems organization~Redundancy</concept_desc>
%%   <concept_significance>300</concept_significance>
%%  </concept>
%%  <concept>
%%   <concept_id>10010520.10010553.10010554</concept_id>
%%   <concept_desc>Computer systems organization~Robotics</concept_desc>
%%   <concept_significance>100</concept_significance>
%%  </concept>
%%  <concept>
%%   <concept_id>10003033.10003083.10003095</concept_id>
%%   <concept_desc>Networks~Network reliability</concept_desc>
%%   <concept_significance>100</concept_significance>
%%  </concept>
%% </ccs2012>
%% \end{CCSXML}

%% \ccsdesc[500]{Computer systems organization~Embedded systems}
%% \ccsdesc[300]{Computer systems organization~Redundancy}
%% \ccsdesc{Computer systems organization~Robotics}
%% \ccsdesc[100]{Networks~Network reliability}

%%
%% Keywords. The author(s) should pick words that accurately describe
%% the work being presented. Separate the keywords with commas.
\keywords{Java, hashCode, mutable classes, runtime verification}

\maketitle

\section{Introduction}

Most mainstream object-oriented languages provide a notion of equality between objects which can be customized to be weaker than
reference equality, and which is coupled with the customizable notion of object hash code \cite{Bloch18}. Such two notions are provided
through two corresponding methods defined in the predefined class \lstinline{Object} which is at the root of the inheritance hierarchy; hence,
they are inherited or can be redefined in any class, and are callable on any type of object.

For this reason, they are pervasive in object-oriented code and the correct functioning of some features in many libraries rely on them;
hence, their incorrect redefinition or use may have a serious impact on software reliability and safety.

A classical example of useful redefinition of equality is for value classes, where typically a notion of logical equality is needed which differs
from reference equality. 
Obeying the general contract for equality is challenging, and equality redefinition invalidates the general contract for computing
object hash codes \cite{Bloch18}.

Indeed, implementations of hash tables typically use equality and object hash codes, therefore
a general contract has to be satisfied: if two objects are equal,
then the same hash code must be computed for them.

If this requirement is not satisfied, then hash tables fail to behave correctly.
Indeed, to find an element in a hash table, its hash code is computed to identify its bucket, then
equality is used to test whether the element is contained in such a bucket. If an equal element is already contained in the hash table, but in
a different bucket, because the computed hash code is different, then the element cannot be found.

While this problem is well known and there have been some attempts to detect it with verification techniques \cite{Bloch18,OkanoHSON19},
hash code redefinition for mutable classes has been overlooked.
When objects of such classes are used as keys in hash tables, programs may exhibit unexpected and unpredictable behavior. Indeed,
if an object is modified while contained in a hash table, then most likely the same object can no longer be found in the table
even though no operations have been performed on the hash table.

Redefinition of equality and hash code in mutable classes is unsafe, as pointed out in
the documentation for \lstinline{java.util.Set} \cite{NelsonEtAl2010} and, similarly, \lstinline{java.util.Map}: 
``\emph{Great care must be exercised if mutable objects are used as set elements. The behavior of a set is not specified if the value of an object is changed in a manner that affects equals comparisons while the object is an element in the set. A special case of this prohibition is that it is not permissible for a set to contain itself as an element.}''

Despite this note, many widely used API libraries do that in Java and other similar languages. 
Verifying that mutable objects with redefined hash code are used correctly in hash tables is not an easy task, because state modification needs to be tracked with a certain precision and rather complex control-oriented properties \cite{AhrendtCPS17,AnconaDF18} have to be ensured.

In this paper we present a solution based on Runtime Verification (RV), a dynamic verification technique where
a single execution of the system under scrutiny (SUS) is abstracted by an event trace
which is checked by a monitor compiled from the formal specification defining the correct behavior
of the SUS.

Events are usually generated by instrumented code of the SUS, and logged or directly sent to the monitor.
Although specification of properties and code instrumentation can be mixed together, decoupling the two activities favors abstraction, reuse and
interoperability of the generated monitors.

Monitors can be \emph{offline} or \emph{online};
in offline RV a trace is typically generated by the instrumented SUS and stored into a log file and then is analyzed by the monitor.
In online RV traces are analyzed real-time to allow error detection to trigger specific actions on the SUS. 
Offline RV \cite{Colombo2022} is a useful solution to integrate other approaches as debugging and testing;  %when errors are not critical or online RV would be too costly;
online RV can be employed to allow error recovery in critical scenarios, providing that such a choice is compatible with the overhead of code instrumentation and of the monitor execution. 

RV is complementary to formal verification and testing:
as formal verification, RV is based on a specification formalism; as happens for software testing, it
scales well to real systems and complex properties, but cannot guarantee exhaustiveness.
Differently from testing, it is particularly useful to ensure control-oriented properties \cite{AhrendtCPS17,AnconaDF18} 
and detect errors due to non-deterministic behavior \cite{havelund2004,sharma2009}.
Furthermore, online monitoring allows runtime contract enforcement, fault protection and automatic program repair.
Finally, several RV tools are based on abstract and intuitive specification languages that can be easily mastered by the user
and favor system agnosticism, portability, reuse, and interoperability.

Our proposed solution is based on offline RV and uses  \rml\footnote{\href{https://rmlatdibris.github.io/}{https://rmlatdibris.github.io/}.}, a rewriting-based Domain Specific Language (DSL) for RV
which allows definition of formal specifications independently of code instrumentation and of the programming language used
to develop the software to be verified. The choice of \rml makes our solution easily portable to different Java-like languages.
Offline RV has been preferred over online RV because the main aim is to detect unsafe use of hash tables; this allows also a simpler solution
which minimize overhead.

The paper is structured as follows.
\Cref{sec:example} introduces the problem in detail and analyzes it in the context of several mainstream object-oriented languages,
\Cref{sec:rml} provides an introduction to \rml, \Cref{sec:spec} presents the proposed solution and discussed possible generalization,
\Cref{sec:rel-concl} is devoted to the related work and conclusions.

\section{Hash code and mutable classes}\label{sec:example}

Correctness issues concerned with the relationship between methods \lstinline{equals} and \lstinline{hashCode} are well-known
in Java \cite{Bloch18,OkanoHSON19} and other object-oriented languages as C\#, Kotlin, Scala, and Python supporting redefinition
of object equality and hash code; however, less attention has been devoted to the potentially dangerous effects of the redefinition of 
\lstinline{hashCode} in mutable classes when their instances are used in container objects implemented with hash tables.

In Java (and Kotlin and Scala as well)  such a problem is more serious because the widely used mutable classes of \lstinline{java.util}
implementing interfaces as \lstinline{Collection} and \lstinline{Map}\footnote{For brevity we refer to types in \lstinline{java.util} with their simple names.}  redefine method \lstinline{hashCode} as their instances where immutable (i.e. value objects).

Let us consider an example, where for simplicity a unique class is used both for containers (which use a hash table) and container elements
since \lstinline{HashTable} is a mutable class redefining \lstinline{hashCode()} and implementing \lstinline{Collection}.
Similar examples can be built with other types of contained elements, for instance, linked lists.
\begin{lstlisting}[numbers=right,numbersep=-7pt]
var sset = new HashSet<Set<Integer>>();
var s = new HashSet<>(asList(1,2,3));
sset.add(s); // sset is {{1,2,3}}
assert sset.contains(s); // success
s.remove(1);
assert sset.contains(s); // failure
s.add(1);
assert sset.contains(s); // success
\end{lstlisting}
Two sets are created with class \lstinline{HashSet}; in such a class, method \lstinline{hashCode} of the elements is used to identify the
bucket where they are stored in the hash table, and method \lstinline{equals} to search them in the bucket.
After execution of the first three lines \lstinline{sset} contains \lstinline{s} as stated by the successful assertion at line 4; in turn, \lstinline{s}
contains the three elements of type \lstinline{Integer} corresponding to 1, 2 and 3.

At line 5 element 1 is removed from \lstinline{s} and at the next line the same assertion is checked again; this time the assertion fails, although no method has been invoked on \lstinline{sset} and, hence, its state should be the same as in the previous assertion. 

This does not come to surprise once one looks at the documentation and discovers that methods \lstinline{equals} and \lstinline{hashCode} are overridden in \lstinline{HashSet}\footnote{Actually, in its direct abstract superclass \lstinline{AbstractSet}.} to depend on all the elements contained in the set. As a consequence, the integer returned by  \lstinline{s.hashCode()} changes after removing element 1 from \lstinline{s} and, hence, the assertion at line 6 fails because \lstinline{s} is searched in the wrong bucket of the hash table of \lstinline{sset}.
As a matter of fact, the assertions at line 4, 6, and 8 depend on the state of both \lstinline{sset} and \lstinline{s}.

What is worst is that the outcome of the assertion at line 6 is unpredictable; indeed, it is still possible, although unlikely,
that the searched bucket is the right one after removing the element from \lstinline{s}. In this case the assertion succeeds.
Finally, considering also that the general contract states that the hash code needs not remain consistent
from one execution of an application to another execution of the same application, we can state that 
the behavior of assertion at line 6 can be non-deterministic.

Once element 1 is inserted back in \lstinline{s}, the computed hash code of the object is again that at line 4, hence assertion at line 8 succeeds. 

Putting it all together, the main source of the problem consists in the fact that in the mutable classes implementing \lstinline{Collection} the receiver in the redefined methods \lstinline{equals} and \lstinline{hashCode} is considered as an immutable object. This should be avoided for all mutable classes whose objects may be used as keys in hash tables, because the consequence is that the object should be ``frozen'' until is no longer in the table to avoid misbehavior as described above. In case an application does not follow this good practice, code should be verified to detect
issues that leads to inconsistencies in hash tables. 

While in C\# and Python it is still possible for the programmers to define mutable classes where the corresponding methods for
equality and hash code are not well-behaved w.r.t. hash tables, predefined mutable collections do not exhibit the problems of Java, Kotlin and Scala.
\begin{lstlisting}
var sset = new HashSet<ISet<int>>();
var s = new HashSet<int>(new int[] { 1, 2, 3 });
sset.Add(s);
Debug.Assert(sset.Contains(s)); // success
s.Remove(1);
Debug.Assert(sset.Contains(s)); // success 
s.Add(1);
Debug.Assert(sset.Contains(s)); // success
\end{lstlisting}
In the C\# code snippet above all assertions succeed simply because methods \lstinline{Equals} and \lstinline{GetHashCode} are not redefined
in mutable classes implementing collections, but inherited from \lstinline{Object}.

Interestingly, in Python for the predefined types \lstinline{set}, \lstinline{list} and \lstinline{dict} another strategy has been adopted: the objects are compared\footnote{In Python object equality can be redefined through method \lstinline{__eq__} to change the behavior of the \lstinline{==} operator.} as
immutable objects, but computing their hash code throws an exception:
\begin{lstlisting}[language=Python]
sset=set()                                                           
s1=set([1,2,3])
s2=set([1,2,3])
assert s1==s2 // success
sset.add(s1) # TypeError: unhashable type: 'set'       
\end{lstlisting}
In this way it is not possible to use sets, lists and dictionaries as hash table keys; this a drastic solution which prevents, for instance, to easily manage sets of sets or lists. 

Finally, JavaScript does not support redefinition of object equality and hash code, hence does not exhibit the issue shown above. 
%% \begin{lstlisting}[language=Javascript]
%% let sset=new Set();                                                           
%% let s=new Set([1,2,3]); // sset is {{1,2,3}}                                                         
%% sset.add(s);
%% console.assert(sset.has(s)); // success
%% s.delete(1);
%% console.assert( sset.has(s)); // success
%% s.add(1);
%% console.assert( sset.has(s)); // success
%% \end{lstlisting}

\section{\rml}
\label{sec:rml}
\rml\cite{RML2021} is a rewriting-based DSL for RV which allows developers to define formal specifications independently of code instrumentation.

It  is based on the notion of \emph{event type} (denoting a set of events) and \emph{trace expression} (denoting a set of event traces),
and it is implemented by a compiler, which generates monitors able to run independently of the SUS and of its instrumentation. 
%The language is based on previous work on RV and global types \cite{CastagnaEtAl12,AnconaBB0CDGGGH16,AnconaFM17},
%applied to several contexts, including verification of interaction protocols in multi-agent systems \cite{AnconaDM12,AnconaBFMT14, BriolaMA14}.

\paragraph{Events}
In \rml an \emph{event} is any observation relevant for monitoring the SUS.
Events are represented in a general way with object literals and
consist of properties which identify the type of event and the data associated with it. For instance,

\begin{lstlisting}
{event:"func_post",targetId:9,name:"add",
    res:true,args:[1]}
\end{lstlisting}          
represents the event
`call to method \lstinline{add} on target object with id 9 and with argument 1  has returned value \lstinline{true}.

An \rml specification defines the set of event traces expected from correct runs of the SUS; the monitor automatically generated from
such a specification checks that the trace generated by a single run of the SUS belongs to such a set.

\paragraph{Event Types}
The basic blocks which constitute an \rml specification are \emph{patterns} built from
\emph{event types} defining sets of events.

Event types are defined with clauses as the following one
\begin{lstlisting}[basicstyle=\ttfamily\scriptsize]
add(id) matches {event:'func_post', targetId:id, name:'add', argIds:[_], res:true};
\end{lstlisting}
The wildcard \lstinline!_! is used when a value is not relevant for
the definition of the event type; in this case \lstinline{add} has to match the
id of the target and the returned result, but not the passed argument.

As a more general example
\begin{lstlisting}[basicstyle=\ttfamily\footnotesize]
modify(id) matches {event:'func_post', targetId:id, name:'add' | 'addAll' | 'remove' | 'removeAll' | 'removeIf' | 'retainAll', argIds:[_], res:true};
\end{lstlisting}
defines the event type matching all calls on target with object identifier id which modify it. 

%\noindent then $\mtch(\ev,\mathit{open}(x))$ successfully returns the substitution
%$\{ x \mapsto 42 \}$.

%%Trace expressions have been extended with variables in \cite{AnconaFM17}.

\rml allows also the definition of event types derived from others, through\footnote{For simplicity, in definitions of derived event types the current implementation of \rml supports only the `not' and `or' operators.} the `or' and `not' operators:
\begin{lstlisting}[basicstyle=\ttfamily\footnotesize]
file_op(fd) matches write(fd) | read(fd);
not_open not matches open(_);
\end{lstlisting}  
The event type pattern \lstinline{file_op(fd)} matches all events corresponding to read or write operations on a file descriptor \lstinline{fd}, and can
be directly derived from the event type patterns \lstinline{write(fd)} and \lstinline{read(fd)}, while \lstinline{not_open} matches all
events not matching \lstinline{open(fd)} for some \lstinline{fd}.

\paragraph{Trace Expressions}
Constitute the basic layer of \rml for defining specifications.
they can be built by combining together event type patterns with primitive and derived operators.
The former kind of operators includes the constant \lstinline{empty}, denoting the singleton set
with the empty trace, the unary postfix operator \lstinline{!} for prefix closure, and the following binary operators:
\begin{itemize}
	%% \item $\emptyseq$ (\emph{empty trace}): the singleton set $\{\emptyseq\}$ containing  the empty event trace $\emptyseq$;
	%% \item $\eventTy\prefixop\tau$ (\emph{prefix}): the set of all traces whose first event $\ev$ matches the event type $\eventTy$, and the remaining part is a trace of $\tau$;
	\item \emph{concatenation} (denoted by juxtaposition) to express sequentiality;
	\item \emph{intersection} \lstinline{/\} for simultaneous verification of multiple properties;
	\item \emph{union} \lstinline{\/} for defining alternatives; 
	\item \emph{shuffle} \lstinline{|} to allow interleaving of events in traces. 
	%% \item $\var{x}{\tau}$ (\emph{binder}): it binds the free occurrences of $\xv$ in $\tau$;
	%% \item $\eventTy\filterop\tau$ (\emph{filter}):
	%% denoting the set of all traces contained in $\tau$, when they are deprived af all events that do not match $\eventTy$ (theoretically, this operator can also be derived from the others).
\end{itemize}
Since trace expressions can be defined recursively, several useful operators can be derived, including the standard postfix operators
\lstinline{?}, \lstinline{+} and \lstinline{*}, borrowed from regular expressions,  the constant \lstinline{all}, which denotes the universe of all traces,
and the  conditional filter operator \lstinline{_ >> _ : _}.
%% which denotes the set of all
%% traces verifying $\te_1$ when only the events matching $\eventTy$ are kept, and
%% $\te_2$ when the events matching $\eventTy$ are filtered out.

%Trace expressions are regular terms (a.k.a. cyclic) \cite{Courcelle83}, thus there is no need for an explicit recursion operator.

For instance, the following simple specification \lstinline{Main} defines a correct usage of a writable file:
\begin{figure}[h]
\begin{lstlisting}[basicstyle=\ttfamily\footnotesize]
open(fd) matches {event:'func_post', name:'fs.openSync', args:[_,'w',...], res:fd};
write(fd) matches {event:'func_pre', name:'fs.writeSync', args:[fd,...]};
close(fd) matches {event:'func_pre', name:'fs.closeSync', args:[fd]};

Main = open(fdesc) write(fdesc)* close(fdesc);
\end{lstlisting}
\caption{Specification of synchronous file operations.}\label{list:sync-fs}
\end{figure}
The specified traces must begin with an event matching the event type pattern \lstinline{open(fdesc)}, that is,
a call to \lstinline{fs.openSync} with returned value matching \lstinline{fdesc}, continue with a possibly empty sequence of events matching
\lstinline{write(fdesc)}, that is, calls\footnote{Events of type  \lstinline{'func_pre'}
correspond to calls to functions that have not returned yet.} to function \lstinline{fs.writeSync} with the first argument matching the same value \lstinline{fdesc}
returned by \lstinline{fs.openSync}, and, finally, must terminate with a call
to function \lstinline{fs.closeSync} with argument matching \lstinline{fdesc} again.

%% Not all operators will be used in this document, but they all can be useful in different contexts.
%% See \cite{ancona2016comparing} for a complete technical presentation of trace expressions with more examples. 

\paragraph{Parametric Specifications}
The expression \lstinline{open(fdesc) write(fdesc)* close(fdesc)} can only verify the correct use of a single file;
the parametric layer of \rml provides a \lstinline{let} construct \cite{AnconaFM17} for declaring variables and delimiting their scope,
to control their dynamic instantiation through event matching.
With such an abstraction and the shuffle operator, it is possible to write a \emph{parametric} specification
for monitoring the access to an arbitrary number of files:
\begin{lstlisting}[basicstyle=\ttfamily\footnotesize]
Main = {let fdesc; open(fdesc) (write(fdesc)* close(fdesc)  | Main)};
\end{lstlisting}
When the first event matches \lstinline{open(fdesc)}, a substitution for variable 
\lstinline{fdesc} is computed, depending from the returned value of \lstinline{fs.openSync}, and such a substitution is applied
to the rest of the specification.
In this way, the two occurrences of \lstinline{fdesc} on the left operand of the shuffle are correctly
instantiated, while the substitution does not affect the right operand, consisting of the recursive
use of \lstinline{Main}, because the nested \lstinline{let} construct allows the declaration of fresh
variables to mask the previously instantiated variables. As a consequence, the
next call to \lstinline{fs.openSync} is allowed to pass a different file name and return a
different file descriptor, because the corresponding event type pattern \lstinline{open(fdesc)}
is not instantiated yet.

%% At this point the shuffle operator is crucial\footnote{In order for the shuffle to work, we assume that every \(\opent\) operation always gives a fresh file descriptor, which can be reasonably assumed to be ensured by the operating system.}: following events are allowed to belong either to \(\tau\) (operations on new files, in the correct order) or to \(\tau'\{\avar{fd}\mapsto x\}\) (note the substitution!) were further operations on \(x\) will be checked.

\paragraph{Generic Specifications}
The most abstract layer of \rml provides support for \emph{generic} specifications,
which enhance modularity, offer more opportunities for reuse and increase the expressive power of \rml with the conditional expression.
For instance, the specification \lstinline{Main} defined above
can be generalized as follows to make it reusable in other specifications:
\begin{lstlisting}[basicstyle=\ttfamily\footnotesize]
File<fdesc> = write(fdesc)* close(fdesc); // only write allowed before closing 
Main = {let fdesc; open(fdesc) (File<fdesc>  | Main)};
\end{lstlisting}
If one would like to change the specification of the allowed operations on the files to be monitored, then only the definition
of the parametric specification \lstinline{File<fdesc>} needs to be redefined:
\begin{lstlisting}[basicstyle=\ttfamily\footnotesize]
...
chmod(fd) matches {event:'func_pre', name:'fs.fchmodSync', args:[fd,_]};

File<fdesc> = chmod(fdesc)? write(fdesc)* close(fdesc); // chmod allowed but only before writing
Main = {let fdesc; open(fdesc) (File<fdesc>  | Main)}; // Main specification unchanged
\end{lstlisting}
Recursion together with the conditional expression are used to define more expressive generic specifications:
\begin{lstlisting}[basicstyle=\ttfamily\footnotesize]
Repeat<n> = if (n>0) eventType Repeat<n-1> else empty;
\end{lstlisting}
The generic specification \lstinline{Repeat<$n$>} defines all traces of length $n$ where all events match
the event type pattern \lstinline{eventType}.


\bibliographystyle{ACM-Reference-Format}
\bibliography{biblio}

\end{document}
\endinput
%%
%% End of file `sample-sigconf.tex'.
