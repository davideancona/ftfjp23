\section{A specification of safe use of collections in hash sets}
\label{sec:spec}
In this section we show how it is possible to define a specification in \rml for dynamically verifying that hash sets and their elements of
type \lstinline{Collection} are managed correctly to avoid the issue highlighted by the examples in \Cref{sec:example}.

The only methods of \lstinline{Collection<E>} that can modify the state of a collection are \lstinline{add(E)} and \lstinline{remove(E)}; other methods, as \lstinline{addAll} and \lstinline{removeAll}, are defined in terms of the primitive ones \lstinline{add} and \lstinline{remove}, hence the specification we consider here covers also them. However, there are additional methods contained in subtypes of collection, consider for instance method
\lstinline{add(int,E)} and \lstinline{remove(int)} of \lstinline{List}, which cannot be monitored through \lstinline{add(E)} and \lstinline{remove(E)}. Possible generalization of the solution presented here are discussed in the last part of this section.

\subsection{Events and Event Types} Since the specification has to verify hash sets, creation of instances of \lstinline{HashTable} is a relevant event
to be monitored.
\begin{lstlisting}[basicstyle=\ttfamily\scriptsize]
new_hash(hash_id) matches {event:'func_post', name:'HashSet', resultId:hash_id};
\end{lstlisting}
Interestingly enough, creation of the elements inserted in the sets need not to be monitored, unless they are hash sets themselves;
as seen in the examples in \Cref{sec:example}, the main point is to trace addition to new elements in sets.

An interesting feature of \lstinline{add} and \lstinline{remove} of \lstinline{Collection} is that they both return true if and only if the operation modifies the collection, hence modifications can be easily monitored at runtime, and it is possible to write a specification based only on events of type \lstinline{'func_post'}.

\begin{lstlisting}[basicstyle=\ttfamily\scriptsize]
add(hash_id,elem_id) matches {event:'func_post', targetId:hash_id, name:'add', argIds:[elem_id], res:true};
remove(hash_id,elem_id) matches {event:'func_post', targetId:hash_id, name:'remove', argIds:[elem_id], res:true};
\end{lstlisting}

After an event matches \lstinline{add(hash_id,elem_id)}, the specification needs to verify that 
element \lstinline{elem_id}, which has just been inserted in the hash set \lstinline{hash_id}, is not modified 
until the element is removed from the set, that is, an event matching \lstinline{remove(hash_id,elem_id)} occurs.
The fact that event type \lstinline{add(hash_id,elem_id)} requires the returned value to be \lstinline{true} is important to avoid
useless checks on elements that are already contained in the set. The same constraint for \lstinline{remove(hash_id,elem_id)} is less important here
because, by construction (see the specification below), the first event matching \lstinline{remove(hash_id,elem_id)} must necessarily be for a call returning value true, assuming correct the implementation of \lstinline{remove}.

The returned value true in the definition of  \lstinline{add(hash_id,elem_id)} and \lstinline{remove(hash_id,elem_id)} is important to avoid false positives when checking that elements in a hash set are not modified. Indeed, the only harmful calls to \lstinline{add} and
\lstinline{remove} are those that effectively change the state of elements and, hence, their hash codes.
\begin{lstlisting}[basicstyle=\ttfamily\scriptsize]
modify(targ_id) matches add(targ_id,_) | remove(targ_id,_);
\end{lstlisting}
There still might be some false positive in (the quite unlikely) case a modification of the element does not change the bucket of the hash table
where it should be contained, as already observed in \Cref{sec:example}. However, this would be hard to be checked and the policy to ban any attempt at modifying elements in a hash set is safer. One might also adopt the stricter policy of prohibiting any call to \lstinline{add} and \lstinline{remove} by omitting the requirement \lstinline{res:true} in the definition of \lstinline{add(hash_id,elem_id)} and \lstinline{remove(hash_id,elem_id)}.

\subsection{Specification}
The whole specification of safe use of collections in hash sets can be found in \Cref{list:hash}.

\begin{figure}[h]
\begin{lstlisting}[basicstyle=\ttfamily\scriptsize]
new_hash(hash_id) matches
  {event:'func_post', name:'HashSet', resultId:hash_id};
not_new_hash not matches new_hash(_);
add(hash_id,elem_id) matches
  {event:'func_post', targetId:hash_id, name:'add',
   argIds:[elem_id], res:true};
not_add(hash_id) not matches add(hash_id,_);
remove(hash_id,elem_id) matches
  {event:'func_post', targetId:hash_id, name:'remove',
   argIds:[elem_id], res:true};
modify(targ_id) matches add(targ_id,_) | remove(targ_id,_);
not_modify_remove(hash_id,elem_id) not matches
  modify(elem_id) | remove(hash_id,elem_id);
op(hash_id,elem_id) matches
  {targetId:hash_id} | {targetId:elem_id};

Main = not_new_hash*
  {let hash_id;new_hash(hash_id)
   (SafeHashTable<hash_id> /\ Main)
  }?;
SafeHashTable<hash_id> = not_add(hash_id)*
  {let elem_id;add(hash_id,elem_id)
   (SafeHashElem<hash_id,elem_id> /\ SafeHashTable<hash_id>)
  }?;
SafeHashElem<hash_id,elem_id> = 
  not_modify_remove(hash_id,elem_id)* (remove(hash_id,elem_id) all)?;
\end{lstlisting}
\caption{Specification of safe hash sets.}\label{list:hash}
\end{figure}

The first part of the specification contains the definitions of all needed event types. The main types have been already introduced,
but there are also some auxiliary types, most of them derived.

The definition of the main specification \lstinline{Main} is recursive, similarly as shown in the example of parametric specification in
\Cref{sec:example}, but the intersection operation is used instead of the shuffle. This is necessary because several hash sets may
coexist and modification of a collection has to be checked for all of them, since such a collection could be contained in any of them.

Before a new hash set is created (event pattern \lstinline{new_hash(hash_id)}), several other events relevant for the generic specifications \lstinline{SafeHashTable} or  \lstinline{SafeHashElem} may occur (trace expression \lstinline{not_new_hash*}).
After a new hash table is created with id \lstinline{hash_id}, the specification \lstinline{SafeHashTable} checks the correct behavior of
the newly created set and \lstinline{Main} manages creation of new hash sets (trace expression \lstinline{SafeHashTable<hash_id> /\ Main}).

For instance, after creation of two hash sets with id 5 and 9, the specification defined by \lstinline{Main} is rewritten into
\begin{lstlisting}[basicstyle=\ttfamily\scriptsize]
(SafeHashTable<5> /\ (SafeHashTable<9> /\ Main)?)?;
\end{lstlisting}
Such a specification represents the current state of the monitor generated from \lstinline{Main}.

The regular expression operator \lstinline{?} (optionality)
is used to cover cases where a specific run of the SUS does not create any hash table.

The definition of \lstinline{SafeHashTable} follows the same pattern as \lstinline{Main}.
Before a new element is added to the hash set \lstinline{hash_id} (event pattern \lstinline{add(hash_id,elem_id)}), several other events relevant for the generic specification \lstinline{SafeHashTable} may occur (trace expression \lstinline{not_add(hash_id)*}).

After a new element with id \lstinline{elem_id} is added, the specification \lstinline{SafeHashElem} checks that \lstinline{elem_id} is not modified
until it is removed from \lstinline{hash_id} and \lstinline{SafeHashTable} manages addition of new elements to the hash sets
(trace expression \lstinline{SafeHashElem<hash_id,elem_id> /\ SafeHashTable<hash_id>}).

\lstinline{SafeHashElem<hash_id,elem_id>} is defined by the trace expression
\begin{lstlisting}
not_modify_remove(hash_id,elem_id)* (remove(hash_id,elem_id) all)?
\end{lstlisting}

It defines the set of traces where modifications on \lstinline{elem_id} are not allowed before an event matching \lstinline{remove(hash_id,elem_id)}
occurs. The pattern \lstinline{not_modify_remove(hash_id,elem_id)} matches all events that do not match \lstinline{modify(elem_id)} and \lstinline{remove(hash_id,elem_id)}. The latter constraint is needed to ensure that removal of \lstinline{elem_id} is checked only once, that is, the corresponding event matches pattern \lstinline{remove(hash_id,elem_id)}. 
The predefined constant \lstinline{all} (the universe of all traces) specifies that no further checks are needed once \lstinline{elem_id} has been removed from \lstinline{hash_id}. The use of the optional operator in \lstinline{(remove(hash_id,elem_id) all)?} reflects the fact that the specification
is used to monitor a safety property: removing \lstinline{elem_id} from \lstinline{hash_id} is a necessary condition for considering safe
all those operations that modify \lstinline{elem_id}, but execution can safely terminate even though \lstinline{elem_id} has not been removed,
if no modification of \lstinline{elem_id} occurred after its insertion in \lstinline{hash_id}.

Reconsidering the rewriting example above, after creation of two hash sets with id 5 and 9, and insertion
of the set with id 9 into the set with id 5, the specification defined by \lstinline{Main} is rewritten into
\begin{lstlisting}[basicstyle=\ttfamily\scriptsize]
((SafeHashElem<5,9> /\ SafeHashTable<5>)? /\ (SafeHashTable<9> /\ Main)?)?;
\end{lstlisting}

\subsection{Possible Generalization of the Specification}

The specification above deals exclusively with objects of type \lstinline{HashSet} for what concerns classes based on hash tables,
and objects of type \lstinline{Collection} for what concerns objects of mutable classes that redefine \lstinline{hashCode}. 

\subsubsection{Classes Based on Hash Tables.}
\lstinline{HashMap} is a widely used class of the Java API. To consider also this class,
some event types, as \lstinline{new_hash} or \lstinline{add}, need to be generalized. 
\begin{lstlisting}[basicstyle=\ttfamily\scriptsize]
new_hash(hash_id) matches {event:'func_post', name:'HashSet' | 'HashMap', resultId:hash_id};
add(hash_id,elem_id) matches // addition to a set
  {event:'func_post', targetId:hash_id, name:'add',  
   argIds:[elem_id], res:true}
 |   // addition to a map  
  {event:'func_post', targetId:hash_id, name:'put',  
   argIds:[elem_id,_], res:null};
\end{lstlisting}
Differently from \lstinline{HashSet}, for some methods it is more challenging to keep exact track of the keys contained in a map, because \lstinline{null} values are allowed. For instance, \lstinline{put(key,value)} returns the previous value associated
with \lstinline{key} (which may include \lstinline{null}) or \lstinline{null} if there was no mapping for \lstinline{key}.
Hence, if we require the result to be \lstinline{null} as done above, then in some cases checking that a key is not modified can be duplicated.
This can be more problematic when hash maps are elements of a hash set (see below).

\subsubsection{Mutable Classes Redefining \lstinline{hashCode}.} 
Package \lstinline{java.util} provides a number of mutable classes where \lstinline{hashCode} is redefined.

The specification above does not take into account that several classes implement subinterfaces of \lstinline{Collection}, such as \lstinline{List}.
For instance, a stack can be modified with methods \lstinline{pop} and \lstinline{push} and the target object is always modified when the method is called.

In most cases the required extension to the specification does not pose any challenge. However, there are some methods, as
\lstinline{set(index,elem)} of \lstinline{List}, for which it is not easy to test whether the target object has been really modified.
For instance, the method
may replace an element in a list with the same object, or an equal object. In that case, the verification detects a false positive.

In \lstinline{java.util} there are also mutable classes redefining \lstinline{hashCode} and implementing interfaces different from
\lstinline{Collection}. We have already considered class \lstinline{HashMap} which implements \lstinline{Map}. In a scenario where
a hash set contains a hash map, modifications on the hash map should be avoided while it is contained in the set.
The \lstinline{put} method above has similar problems as method \lstinline{set}, hence the verification may be less accurate in this case.

\subsubsection{Preliminary Experiments.}
We have conducted preliminary experiments to test the correctness of the specification
with the offline monitor generated from it by the \rml compiler.
The monitor has been run on event traces that simulate the  execution of simple Java programs, as shown below, and that have been stored in log files.
\begin{lstlisting}[numbers=left]
var sset = new HashSet<Set<Integer>>();
var s1 = new HashSet<Integer>();
var s2 = new HashSet<Integer>();
s1.add(1);
s2.add(2);
sset.add(s1);
s1.contains(1);
s1.add(1);
sset.add(s2);
sset.remove(s1);
//s2.remove(2);
s1.remove(1);
s2.remove(1);
sset.remove(s2);
s1.add(1);
s2.add(2);
\end{lstlisting}
The only critical instruction has been commented.
Indeed, at line 11 an element of \lstinline{s2} is removed, while \lstinline{s2} is in the hash set \lstinline{sset}.

With line 11 commented, the corresponding trace is accepted by the monitor, as expected.
In particular, the monitor recognizes that the two methods \lstinline{contains} and \lstinline{add} called on \lstinline{s1}
while contained in \lstinline{sset} (lines 7 and 8) are safe because do not change the state of \lstinline{s1}. Similarly, line 13 for
\lstinline{s2}.

Line 12 is safe, although \lstinline{s1.remove(1)} changes the state of the object, because \lstinline{s1} no longer belongs to
\lstinline{sset}; the same consideration for \lstinline{s1} applies to line 15 and line 16 for \lstinline{s2}. 

If the comment at line 13 is removed, then the corresponding trace is rejected.
%% \begin{lstlisting}[language={},basicstyle=\ttfamily\scriptsize]
%% unmatched event #58:
%%   _36564{argIds:[null],args:[2],event:func_post,name:remove,
%%          res:true,targetId:13}
%% \end{lstlisting}
%% Event number 58 corresponds to line 11, and 13 is the id of the object referenced by \lstinline{s2}.   
