\section{A specification of safe use of collections in hash sets}
In this section we show how it is possible to define a specification in \rml for dynamically verifying that hash sets and their elements of
type \lstinline{Collection} are managed correctly to avoid the issue highlighted by the examples in the previous section.

The only methods of \lstinline{Collection<E>} that can modify the state of a collection are \lstinline{add(E)} and \lstinline{remove(E)}; other methods, as \lstinline{addAll} and \lstinline{removeAll}, are defined in terms of the primitive ones \lstinline{add} and \lstinline{remove}, hence the specification we consider here covers also them. However, there are additional methods contained in subtypes of collection, consider for instance method
\lstinline{add(int,E)} and \lstinline{remove(int)} of \lstinline{List}, which cannot be monitored through \lstinline{add(E)} and \lstinline{remove(E)}. Possible generalization of the solution presented here are discussed in the last part of this section.

\paragraph{Events and event types} An interesting feature of \lstinline{add} and \lstinline{remove} is that they both return true if and only if the operation modifies the collection, hence modifications can be easily monitored at runtime, and it is possible to write a specification based only on events of type \lstinline{'func_post'}.

\begin{lstlisting}[basicstyle=\ttfamily\scriptsize]
add(hash_id,elem_id) matches {event:'func_post', targetId:hash_id, name:'add', argIds:[elem_id], res:true};
remove(hash_id,elem_id) matches {event:'func_post', targetId:hash_id, name:'remove', argIds:[elem_id], res:true};
\end{lstlisting}

After an event matches \lstinline{add(hash_id,elem_id)} the specification needs to verify that 
element \lstinline{elem_id}, which has just been inserted in the hash set \lstinline{hash_id}, is not modified 
until the element is removed from the set, that is, an event matching \lstinline{remove(hash_id,elem_id)} occurs.


\begin{figure}[h]
\begin{lstlisting}[basicstyle=\ttfamily\scriptsize]
new_hash(hash_id) matches {event:'func_post', name:'HashSet', resultId:hash_id};
not_new_hash not matches new_hash(_);
modify(targ_id) matches {event:'func_post', targetId:targ_id, name:'add' | 'remove', res:true};
not_modify_remove(hash_id,elem_id) not matches remove(hash_id,elem_id) | modify(elem_id);
add(hash_id,elem_id) matches {event:'func_post', targetId:hash_id, name:'add', argIds:[elem_id], res:true};
not_add(hash_id) not matches add(hash_id,_);
remove(hash_id,elem_id) matches {event:'func_post', targetId:hash_id, name:'remove', argIds:[elem_id], res:true};
op(hash_id,elem_id) matches {targetId:hash_id} | {targetId:elem_id};
relevant matches new_hash(_) | modify(_);

Main = not_new_hash* {let hash_id;new_hash(hash_id) (SafeHashTable<hash_id> /\ Main)}?;
SafeHashTable<hash_id> = not_add(hash_id)* {let elem_id;add(hash_id,elem_id) (SafeHashElem<hash_id,elem_id> /\ SafeHashTable<hash_id>)}?;
SafeHashElem<hash_id,elem_id> = op(hash_id,elem_id) >> not_modify_remove(hash_id,elem_id)* remove(hash_id,elem_id) all;
\end{lstlisting}
\caption{Specification of safe hash setd.}\label{list:hash}
\end{figure}
