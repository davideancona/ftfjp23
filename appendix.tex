\appendix
\section{Example of verified Java code}\label{appendix}

A simple Java snippet verified by the \rml specification defined in \Cref{list:hash}.
\begin{lstlisting}[numbers=left]
var sset = new HashSet<Set<Integer>>();
var s1 = new HashSet<Integer>();
var s2 = new HashSet<Integer>();
s1.add(1);
s2.add(2);
sset.add(s1);
s1.contains(1);
s1.add(1);
sset.add(s2);
sset.remove(s1);
//s2.remove(2);
s1.remove(1);
s2.remove(1);
sset.remove(s2);
s1.add(1);
s2.add(2);
\end{lstlisting}
If line 11 is commented, then the code is correct w.r.t. the specification and the monitor successfully analyzes the whole trace generated by the execution of the snippet, consisting of 82 events stored in a log file:
\begin{lstlisting}[language={},basicstyle=\ttfamily\scriptsize]
...
matched event #82: _21156{argIds:[null],args:[2],event:func_post,name:add,res:false,targetId:13}
Execution terminated correctly
\end{lstlisting}

If line 11 is uncommented, then the code is not correct w.r.t. the specification because set \lstinline{s2} while
contained in the hash set \lstinline{sset}.

The monitor reports an error for event number 58 corresponding to the return from method call
at line 11, where 13 is the id of the object referenced by \lstinline{s2}:
\begin{lstlisting}[language={},basicstyle=\ttfamily\scriptsize]
unmatched event #58:
  _36564{argIds:[null],args:[2],event:func_post,name:remove,
         res:true,targetId:13}
\end{lstlisting}
