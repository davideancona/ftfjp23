\section{Problems with hashCode and mutable classes}

While correctness issues concerned with the relationship between methods \lstinline{equals} and \lstinline{hashCode} are well-known
in Java and C\# \cite{Bloch18,OkanoHSON19}, less attention is devoted to the potentially dangerous effects of the redefinition of 
\lstinline{hashCode} in mutable classes when their instances are used in container objects implemented with hash tables.

In Java such a problem is more serious because all mutable classes implementing interface \lstinline{java.util.Collection} redefine
method \lstinline{hashCode} as their instances where immutable (i.e. value objects).

Let us consider this simple example:
\begin{lstlisting}[numbers=left]
var sset = new HashSet<Set<Integer>>();
var s = new HashSet<>(asList(1,2,3));
sset.add(s); // sset is {{1,2,3}}
assert sset.contains(s); // success
s.remove(1);
assert sset.contains(s); // failure
s.add(1);
assert sset.contains(s); // success
\end{lstlisting}
Two sets are created with class \lstinline{java.util.HashSet}; in such a class method \lstinline{hashCode} of the elements is used to identify the
bucket where they are stored in the hash table, and method \lstinline{equals} to search them in the bucket.
After execution of the first three lines \lstinline{sset} contains \lstinline{set} as stated by the succesfull assertion at line 4; in turn, \lstinline{set}
contains the three elements of type \lstinline{Integer} corresponding to 1, 2 and 3.

At line 5 element 1 is removed from \lstinline{set} and at the next line the same assertion is checked again; this time the assertion fails, although no method has been invoked on \lstinline{sset} and, hence, its state should be the same as in the previous assertion.


\begin{lstlisting}
var sset=new HashSet<ISet<int>>();
var set=new HashSet<int>(new int[]{1,2,3});
sset.Add(set);
Debug.Assert(sset.Contains(set));
set.Remove(1);
Debug.Assert(sset.Contains(set));
set.Add(1);
Debug.Assert(sset.Contains(set));
\end{lstlisting}


\begin{lstlisting}[language=Python]
sset=set()                                                           
s=set([1,2,3])                                                         
sset.add(s) # TypeError: unhashable type: 'set'       
\end{lstlisting}

\begin{lstlisting}[language=Javascript]
let sset=new Set();                                                           
let s=new Set([1,2,3]); // sset is {{1,2,3}}                                                         
sset.add(s);
console.assert(sset.has(s)); // success
s.delete(1);
console.assert( sset.has(s)); // success
s.add(1);
console.assert( sset.has(s)); // success
\end{lstlisting}




