\documentclass[10pt,usenames,dvipsnames]{beamer}

\usepackage{amsmath}
\usepackage{stmaryrd}
\usepackage{centernot}
\usepackage{xspace}
\usepackage{listings}
\usepackage{hyperref}
\usepackage{xcolor}
\usepackage{graphicx}
\usepackage{macros}

%% Angelo's packages

\usepackage[utf8]{inputenx} % For æ, ø, å
\usepackage{csquotes}       % Quotation marks
\usepackage{microtype}      % Improved typography
\usepackage{amssymb}        % Mathematical symbols
\usepackage{mathtools}      % Mathematical symbols
\usepackage{stmaryrd}
\usepackage[absolute, overlay]{textpos} % Arbitrary placement
\setlength{\TPHorizModule}{\paperwidth} % Textpos units
\setlength{\TPVertModule}{\paperheight} % Textpos units
\usepackage{tikz}
\usetikzlibrary{overlay-beamer-styles}  % Overlay effects for TikZ
\usepackage{graphicx}
\usepackage{array}
\usepackage{amssymb}
\usepackage{stmaryrd}
\usepackage{subcaption}
\usepackage{float}
\usepackage{url}
\usepackage{doi}
\usepackage[all]{xy}

\lstset{language=Java,basicstyle=\ttfamily\footnotesize,commentstyle=\itshape,morekeywords={assert},keywordstyle=\ttfamily\bfseries}

%\usetheme[secfooter]{PaloAlto}
%\usetheme[secfooter]{Hannover}
%\usetheme[]{CambridgeUS}
\usetheme[secfooter]{UiB}

\usebackgroundtemplate{\hspace*{.85\textwidth}\includegraphics[keepaspectratio,width=.2\textwidth]{images/logoBW.png}}

\title[Runtime Verification of Hash Code in Mutable Classes]{Runtime Verification of Hash Code in Mutable Classes} 

\author[D. Ancona]{\underline{Davide Ancona}, Angelo Ferrando and \\ Viviana Mascardi \\[1ex]
  \small DIBRIS, Universit\`a di Genova, Italy}

\institute[]{DIBRIS, Universit\`a di Genova, Italy}
\date[FTfJP23]{Formal Techniques for Java-like Languages, July 18, 2023}
\subject{}

%% \AtBeginSection[]
%% {
%%     \begin{frame}
%%         \frametitle{Table of Contents}
%%         \tableofcontents[currentsection]
%%     \end{frame}
%% }

\begin{document}

%%\frame{\titlepage}

\begin{frame}{Outline}
\tableofcontents%[currentsection]
\end{frame}

\section{Object equality and hash code}

\begin{frame}{Contract for equality and hash code}
  \begin{block}{Features of most object-oriented languages}
    \begin{itemize}
    \item two different notions of equality
      \begin{itemize}
      \item by reference, predefined (\lstinline{==})
      \item weaker equality, user-defined (\lstinline{equals})
      \end{itemize}
    \end{itemize}
  \end{block}

  \begin{block}{General contract in \lstinline{java.lang.Object}}
    \emph{If two objects are equal, then the same hash code must be computed for them}
  \end{block}

    \begin{block}{Reason}
      Classes as \lstinline{HashSet} or \lstinline{HashMap} rely on \lstinline{equals} and \lstinline{hashCode}:
      \begin{itemize}
      \item \lstinline{hashCode} is used to retrieve a specific bucket
      \item \lstinline{equals} is used to find an element in such a specific bucket        
        \end{itemize}
  \end{block}

\end{frame}

%%%%%%%%%%%%%%%%%%%%%%%%%%%%%%%%%%%%%%%%%%%%%%%%%%

\begin{frame}{Contract for equality and hash code}
  \begin{block}{A stricter contract in \lstinline{java.util.Set}}
    \emph{Great care must be exercised if mutable objects are used as set elements.}\\[1ex]
    \emph{The behavior of a set is not specified if the value of an object is changed in a way that affects equals comparisons while the object is an element in the set.} \\[1ex]
    \emph{A special case of this prohibition is that it is not permissible for a set to contain itself as an element.} 
  \end{block}
\end{frame}

%%%%%%%%%%%%%%%%%%%%%%%%%%%%%%%%%%%%%%%%%%%%%%%%%%

\begin{frame}[fragile]{Contract for equality and hash code}
  \begin{block}{A simple example}
    \begin{lstlisting}[language=Java]
var sset = new HashSet<Set<Integer>>();
var s = new HashSet<>(asList(1)); // s is {1}
sset.add(s); // sset is {{1}}
assert sset.contains(s); 
s.remove(1);
assert sset.contains(s); 
s.add(1);
assert sset.contains(s); 
    \end{lstlisting}
  \end{block}
\end{frame}

%%%%%%%%%%%%%%%%%%%%%%%%%%%%%%%%%%%%%%%%%%%%%%%%%%

\begin{frame}[fragile]{Contract for equality and hash code}
  \begin{block}{A simple example}
    \begin{lstlisting}[language=Java]
var sset = new HashSet<Set<Integer>>();
var s = new HashSet<>(asList(1)); // s is {1}
sset.add(s); // sset is {{1}}
assert sset.contains(s); // success
s.remove(1);
assert sset.contains(s); // failure
s.add(1);
assert sset.contains(s); // success
    \end{lstlisting}
  \end{block}
\end{frame}

%%%%%%%%%%%%%%%%%%%%%%%%%%%%%%%%%%%%%%%%%%%%%%%%%%

\begin{frame}[fragile]{Contract for equality and hash code}
  \begin{block}{Another example}
    \begin{lstlisting}[language=Java]
var sset = new HashSet<Set<Integer>>();
var s = new HashSet<>(asList(0)); // s is {0}
sset.add(s); // sset is {{0}}
assert sset.contains(s); 
s.remove(0);
assert sset.contains(s); 
s.add(0);
assert sset.contains(s); 
    \end{lstlisting}
  \end{block}
\end{frame}

%%%%%%%%%%%%%%%%%%%%%%%%%%%%%%%%%%%%%%%%%%%%%%%%%%

\begin{frame}[fragile]{Contract for equality and hash code}
  \begin{block}{Another example}
    \begin{lstlisting}[language=Java]
var sset = new HashSet<Set<Integer>>();
var s = new HashSet<>(asList(0)); // s is {0}
sset.add(s); // sset is {{0}}
assert sset.contains(s); // success
s.remove(0);
assert sset.contains(s); // success!
s.add(0);
assert sset.contains(s); // success
    \end{lstlisting}
  \end{block}

  \begin{block}{Issues}
    \begin{itemize}
    \item almost unpredictable code behavior
    \item non-deterministic behavior if \lstinline{hashCode} depends on object references
      \begin{itemize}
      \item object references may change from one execution to another 
      \item hash code needs not remain consistent from one execution to another 
      \end{itemize}    
    \end{itemize}    
  \end{block}
\end{frame}

%%%%%%%%%%%%%%%%%%%%%%%%%%%%%%%%%%%%%%%%%%%%%%%%%%

\begin{frame}[fragile]{Theory versus practice}
  \begin{block}{Theory}
    \begin{itemize}
    \item mutable classes should not redefine \lstinline{equals}
    \item weaker contract: \lstinline{hashCode} should not depend on ``mutable'' fields
    \end{itemize}   
  \end{block}

  \begin{block}{Practice}
    \begin{itemize}
    \item mutable classes of \lstinline{java.util.Collection} do not satisfy such a contract 
    \item similar problems in Kotlin and Scala, but not in C\#
    \end{itemize}   
  \end{block}


    \begin{block}{Aims}
    \begin{itemize}
    \item verify that \lstinline{Collection} objects are not modified while in a hash table
    \item proposed solution: Runtime Verification (RV)
    \end{itemize}   
  \end{block}

\end{frame}

%%%%%%%%%%%%%%%%%%%%%%%%%%%%%%%%%%%%%%%%%%%%%%%%%%

\lstset{
	morekeywords={not,false,true,none,any,empty,matches,all,let,if,else,with,abs},
	keywordstyle=\color{blue},
	morestring=[b]',
	stringstyle=\color{violet},
	morecomment=[l]{//},
	commentstyle=\color{cyan},
	mathescape=true,
	basicstyle=\ttfamily,
	captionpos=b,
	tabsize=4,
	breaklines,
	breakatwhitespace,
	showstringspaces=false,
	keepspaces
}

\lstset{literate={\\/}{$\orop$}2 {/\\}{$\andop$}2 {|}{$\shuffleop$}1 {!}{$\closop$}1 {>>}{$\filterop$}2 {:}{$:$}1}

\section{Runtime Verification and \rml}

\begin{frame}{Runtime verification}

  \begin{block}{Definition}
    Runtime Verification (RV) is a \hl{verification technique} that allows for checking whether a \hl{run} of a system under scrutiny (SUS) \hl{satisfies or violates} a given correctness property.
  \end{block}

  \begin{block}{Main ingredients}
   \begin{itemize}
   \item run = possibly infinite event trace
   \item instrumentation = generates the relevant events
   \item formal specification = a set of event traces     
   \item monitor =  generated from a specification, dynamically checks finite prefixes of a run 
\end{itemize}
  \end{block}

\end{frame}

\begin{frame}{Runtime verification}

  \begin{center}
    \includegraphics[keepaspectratio,height=0.7\textheight]{images/rv}
  \end{center}

\end{frame}

\begin{frame}{Why RV?}
  \begin{block}{RV bridges the gap between formal verification and testing}
    \begin{itemize}
    \item the notion of event trace abstracts over system runs
    \item RV offers error recovery, self-adaptation, and issues that go beyond software reliability
    \item some information available only at runtime
    \item the behavior of an application may depend heavily on the environment of the target system
    \item if security is an issue, RV provides cast-iron guarantees for properties that have been statically proved or tested
    \end{itemize}
  \end{block}
\end{frame}

\begin{frame}{RV taxonomy}
    \begin{figure}
        \centering
        \includegraphics[width=0.59\linewidth]{images/example6.png}
    \end{figure}
\end{frame}

\begin{frame}{RV taxonomy}
  \begin{block}{Stages of monitoring}
    \begin{itemize}
    \item \hl{on line}
      \begin{itemize}
      \item during execution of the system
      \item storing of the trace unnecessary
      \item      supports runtime enforcement
      \end{itemize}
    \item \hl{off line}
      \begin{itemize}
      \item  after execution of the system, with the trace stored in log files
      \item      does not support runtime enforcement
      \item complementary to testing and debugging
      \end{itemize}
    \end{itemize}
  \end{block}

\end{frame}


\begin{frame}{\rml}
  \begin{center}
    \includegraphics[height=0.7\textheight]{images/rmlweb}
  \end{center}

  \rml Web page: \href{https://rmlatdibris.github.io/}{https://rmlatdibris.github.io/}
\end{frame}

\begin{frame}{\rml}

  \begin{block}{Main aims of \rml}
    \begin{itemize}
    \item \hl{formal language approach}: extension of deterministic CF grammars 
    \item \hl{usability}: developers are familiar with regular expressions and grammars
    \item \hl{expressive power}: more expressive than deterministic CF grammars
    \item \hl{interoperability}: system agnostic, events in independent format
    \end{itemize}
  \end{block}
\end{frame}

\begin{frame}{\rml}
  \begin{block}{Structure of \rml specifications}
    Four different layers
    \begin{itemize}
    \item \hl{event types}: relevant events 
    \item \hl{trace expressions}: primitive and derived operators for defining sets of event traces
    \item \hl{parametricity}: existential quantification w.r.t. data carried by events
    \item \hl{genericity}: specification with parameters to enhance modularity, reuse and expressive power
    \end{itemize}
  \end{block}
\end{frame}

\begin{frame}{Q\&A time}
\vspace*{16ex}
  \begin{center}
  \LARGE Thank you!
\end{center}
\end{frame}

\end{document}
